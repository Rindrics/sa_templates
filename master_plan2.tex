
\documentclass[A4j, disablejfam, titlepage, openright, twoside, 10.5pt]{myjsbook}
\linespread{1.5}
\usepackage[dvipdfmx, hiresbb]{graphicx, xcolor}
\usepackage{draftwatermark}
\usepackage{otf}
\usepackage{array, booktabs}
\usepackage{blindtext}
\usepackage{colortbl}
\usepackage{atbegshi}
\usepackage{amsmath}
\usepackage{url}
\usepackage{fancyhdr}
\usepackage{subfiles}
\usepackage{TW_Engraulis-japonicus}
\pagestyle{fancy}
\renewcommand{\headrulewidth}{0.0pt}
\renewcommand{\chaptermark}[1]{\markright{}}
\renewcommand{\sectionmark}[1]{\markright{}}
\chead{\thechapter--\thepage--}
\cfoot{\normalfont\hfil-- \thepage\ --\hfil}
\ifnum 42146=\euc"A4A2
  \AtBeginShipoutFirst{\special{pdf:tounicode EUC-UCS2}}
\else
  \AtBeginShipoutFirst{\special{pdf:tounicode 90ms-RKSJ-UCS2}}
\fi
%\usepackage{footnpag}
%\makeatletter
%\let\original@@makefntext\@makefntext
%\let\original@footnoterule\footnoterule
%\makeatother
%\usepackage{ftnright}
%\makeatletter
%\let\@makefntext\original@@makefntext
%\let\footnoterule\original@footnoterule
%\makeatother

\makeatletter
\newcommand\footnoteref[1]{\protected@xdef\@thefnmark{\ref{#1}}\@footnotemark}
\makeatother

\usepackage[dvipdfmx,
bookmarks=true,%
bookmarksnumbered=true,%
bookmarksopen=true,%
]{hyperref}
%\setlength{\textwidth}{\fullwidth}
\setlength{\textwidth}{15truecm}
\setlength{\hoffset}{0pt}
\setlength{\voffset}{-1.54truecm}
\setlength{\headsep}{2truecm}
\setlength{\footskip}{2truecm}
\setlength{\textheight}{23truecm}
%\setlength{\textwidth}{35zw}
%\setlength{\textheight}{50zw}
%\setlength{\headwidth}{1truemm}
%\addtolength{\textheight}{\topskip}
%\setlength{\topmargin}{0pt}
%\setlength{\voffset}{-15truemm}
%\setlength{\hoffset}{-5truemm}
\setlength{\marginparsep}{1truecm}
\setlength{\marginparwidth}{0pt}
\setlength{\oddsidemargin}{0.46truecm}
\setlength{\evensidemargin}{\oddsidemargin}
\setlength{\abovecaptionskip}{-3truemm}
\setlength{\belowcaptionskip}{-2truemm}
\SetWatermarkLightness{0.9}
\setcounter{tocdepth}{1} %1だとchapterまで目次に表示される
%\usepackage{ctable, dcolumn}
\usepackage[format=hang, labelsep=period]{caption}
\usepackage[sectionbib]{chapterbib}
\usepackage{float}
\usepackage[T1]{fontenc}
%\usepackage[scaled]{helvet}
\usepackage{layout}
\usepackage{pdflscape}
%\usepackage{myCitation}
\usepackage{multirow}
\usepackage{multicol}
\usepackage{newtxtext, newtxmath}
\usepackage{otf}
\usepackage{pifont}
%\usepackage{subcaption}
\usepackage{subfig}
\usepackage{tabularx}
\usepackage{textcomp}
\usepackage{threeparttable}
\usepackage{subfiles} % 独立タイプセットを可能にするが、chapterbibと併用できないので使用を停止。併用のためには正規表現などで工夫する必要がある。
\usepackage{tocbibind}
\usepackage[authoryear, round, sort&compress]{mynatbib} %natbibを使うと引用文献が続きになってしまう。
%\bibliographystyle{mynatbib}
%\usepackage{jecon}
\bibliographystyle{jecon}     %jeconの利用を検討中。\citealtなどの引用文を変更する必要あり
%\setcitestyle{number,close={)}}

%ここから 引用文献レベルをsubsectionに再定義
\makeatletter
\renewenvironment{thebibliography}[1]{%
  %\global\let\presectionname\relax
  %\global\let\postsectionname\relax
  \subsection{\bibname}\@mkboth{\bibname}{}%
  %\addcontentsline{toc}{subsection}{\bibname}%   これをオンにすると目次に2回出てきてしまう
  \list{\@biblabel{\@arabic\c@enumiv}}%
        {\settowidth\labelwidth{\@biblabel{#1}}%
        \setlength{\itemindent}{-2zw}%      文献リストのインデント
        \setlength{\topsep}{-10zw}%         引用文献」から最初の文献までの空白
        %\leftmargin\labelwidth
        %\advance\leftmargin\labelsep
        %\@openbib@code
        %\usecounter{enumiv}%
        %\let\p@enumiv\@empty
        %\renewcommand\theenumiv{\@arabic\c@enumiv}
        }%
  %\sloppy
  %\clubpenalty4000
  %\@clubpenalty\clubpenalty
  %\widowpenalty4000%
  \sfcode`\.\@m}
  {\def\@noitemerr
    {\@latex@warning{Empty `thebibliography' environment}}%
  \endlist}
\makeatother
%ここまで 引用文献レベルをsubsectionに再定義


%ハイフネーション設定
\hyphenation{Engraulis}
\hyphenation{japonicus}

\renewcommand{\sfdefault}{phv}
\renewcommand{\rmdefault}{qtm}
\renewcommand{\baselinestretch}{1}
\renewcommand{\thefootnote}{*\roman{footnote}}
\renewcommand{~}{\phantom{0}}
\makeatletter
   \renewcommand{\thefigure}{\arabic{figure}}%      図番号の体裁
  \@addtoreset{figure}{section}
 \makeatother

 \makeatletter
 \renewcommand{\thetable}{arabic{table}}%       表番号の体裁
  \@addtoreset{table}{section}
\makeatother
%コマンド
\setcounter{secnumdepth}{3}
\newcommand{\fcur}{$F_{current}$}
\newcommand{\fmed}{Fmed}
\newcommand{\要約}{\begin{center}\Large{{\gt 要   約}}\end{center}}
\newcommand{\digest}{}
%\newcommand{\}{}

%環境
\newenvironment{OneOfSixFigs}[3]{
\begin{figure}[htp]
    \captionsetup{width=65mm}
 \begin{minipage}{0.5\hsize}
  \begin{center}
   \includegraphics[width=70mm]{#1}
  \end{center}
  \caption{#2}
  \label{#3}
 \end{minipage}
\end{figure}}

\newenvironment{TwoOfSixFigs}[6]{
\begin{figure}[htp]
    \captionsetup{width=65mm}
 \begin{minipage}{0.5\hsize}
  \begin{center}
   \includegraphics[width=70mm]{#1}
  \end{center}
  \caption{#2}
  \label{#3}
 \end{minipage}
 \begin{minipage}{0.5\hsize}
  \begin{center}
   \includegraphics[width=70mm]{#4}
  \end{center}
  \caption{#5}
  \label{#6}
 \end{minipage}
\end{figure}}

\newenvironment{TwoOfEightFigs}[6]{
\begin{figure}[htp]
    \captionsetup{width=60mm}
 \begin{minipage}{0.5\hsize}
  \begin{center}
   \includegraphics[width=60mm]{#1}
  \end{center}
  \caption{#2}
  \label{#3}
 \end{minipage}
 \begin{minipage}{0.5\hsize}
  \begin{center}
   \includegraphics[width=60mm]{#4}
  \end{center}
  \caption{#5}
  \label{#6}
 \end{minipage}
\end{figure}}

%担当機関
\newenvironment{担当機関等}[3]{
\begin{table}[h]
\begin{tabular}{{rp{12.2cm}}}
責任担当水研: &#1 (#2)\\
参画機関: & #3
\end{tabular}
\end{table}
}

%過去5年間の資源量等
\newenvironment{過去五年間の資源量等}[5]{
\begin{center}
\begin{tabular}{cccccc}
\toprule
\multirow{2}{*}{年} & {資源量}  & {親魚量}  & {漁獲量}  & {$F$} & {漁獲割合}\tabularnewline
          &   (千トン)  & (千トン)    & (千トン)&   & (\%)  \tabularnewline
\hline
#1  \tabularnewline
#2  \tabularnewline
#3  \tabularnewline
#4  \tabularnewline
#5  \tabularnewline
\bottomrule
\end{tabular}
\end{center}
}

\begin{document}
\frontmatter
%\layout
%
%
\title{
{\thisyrjp}年度
\\
{\HUGE 我が国周辺水域の漁業資源評価}}
\author{\\
\\
\\
\\
\\
\\
\\
\\
\\
{\Large 水産庁増殖推進部}
\\
}
\date{\today}
%\maketitle
%
%
%
\mainmatter
%\part*{第1分冊}
\tableofcontents
%\include{_PF_Sardinops-melanostictus}
%\include{_SI_Sardinops-melanostictus}
%%編集不要。
\chapter{\thisyrjp(\thisyrad)年度マイワシ対馬暖流系群の資源評価}
%
\担当機関等
{西海区水産研究所}%担当水研を入力
{安田十也、黒田啓行、林 晃、依田真里、\CID{8705}橋素光}%担当者名を入力
{日本海区水産研究所、青森県産業技術センター水産総合研究所、秋田県農
林水産技術センター水産振興センター、山形県水産試験場、新潟県水産海
洋研究所、富山県農林水産総合技術センター水産研究所、石川県水産総合
センター、福井県水産試験場、京都府農林水産技術センター海洋センター、
兵庫県立農林水産技術総合センター但馬水産技術センター、鳥取県水産試
験場、島根県水産技術センター、山口県水産研究センター、福岡県水産海
洋技術センター、佐賀県玄海水産振興センター、長崎県総合水産試験場、
熊本県水産研究センター、鹿児島県水産技術開発センター
}%関連機関を入力

\要約
本系群の資源量について、資源量指標値を考慮したコホート解析により求めた。
資源量は、1970年代から増加し、1988年には1千万トンに達したと推定されるが、
1990年代に急減した。2001--2003年に過去最低の水準で推移し、2004年以降は増加傾向にある。
2015年の資源量は298千トンで、親魚量は192千トンである。
2015年の親魚量がBlimit(100千トン)を上回っていることから資源水準は中位で、
最近5年間(2011--2015年)の資源量の推移から動向は横ばいと判断した。
今後、再生産成功率(加入量÷親魚量)が、不確実性の高い直近年(2015年)を除く過去10年
(2005--2014年)の中央値で継続した場合に、
現状の漁獲圧の維持(Fcurrent)、親魚量の増大(F40\%SPR)および親魚量の維持(Fmed≒F30\%SPR)の各漁獲シナリオで期待される漁獲量を2017年ABCとして算定した。

%\input{output/youyaku_table}%表を読み込み

Targetは資源変動やデータ誤差に起因する評価の不確実性を考慮し、各漁獲シナリオの下で、より安定的な資源の増大または維持が期待されるF値による漁獲量で、Limitは各漁獲シナリオの下で許容される最大のF値による漁獲量である。Ftarget = α Flimitとし、係数αには標準値0.8を用いた。Fcurrentは2006--2015年のFの平均値、漁獲割合は2017年の漁獲量/資源量、F値は各年齢の平均値である。漁獲シナリオにある「親魚量の維持」は中長期的に安定する親魚量の維持を指す。{\thisyrad}年の親魚量は192千トン。

\subsection{まえがき}
我が国周辺に分布するカタクチイワシは、太平洋系群、瀬戸内海系群および対馬暖流系群から構成される。
本種の漁獲量は、マイワシとは対照的に1990年代に増加した。対馬暖流域においても、1990年代後半にかけて漁獲量が増加したが、
2001年に急減し、その後は増減を繰り返している。しかし、本種の漁獲量の変動幅はマイワシほど大きくない。
これは、マイワシと比較して親魚になるまでの期間が短いことや、ほぼ周年にわたり産卵を行うことなどが要因と考えられる。

東シナ海や日本海に分布するカタクチイワシは、韓国や中国によっても漁獲されているが、これらの主な分布域は韓国と中国の沿岸域であるため、
対馬暖流系群とはみなさず、本資源評価では考慮しなかった。

\subsection{生態}                 %読み込み専用。
\subsubsection{分布・回遊}
東シナ海北部から九州沿岸(西岸)、日本海にかけて広く分布する。漁獲量が多かった1980年代には沖合域にも分布が見られたが(\citealt{Hiyama1998})、2000年以降はほぼ沿岸域に限られると考えられており(図1)、資源量や生息環境の変化とともに分布域が変化すると考えられる。また、マイワシは主に春と秋に多く漁獲される傾向があり、漁獲量には季節変化がみられる。漁獲量のピークを迎える季節は地域によって異なることから、マイワシは分布域内を大小さまざまな規模で季節回遊しているものと考えられる(伊藤 1961, 黒田 1991)。
\subsubsection{年齢・成長}
\digest{本系群の成長様式は、発生時期によって異なることが知られている。
本報告では、耳石に形成される日周輪の解析結果および体長組成の経月変化から、孵化した個体が半年後には被鱗体長で約9 cmまで成長すると仮定した。}
体長組成の経月変化から、春季と秋季の発生群について成長様式を求めたところ、次のような結果を得た(図\ref{fig_seichou}、\citealt{Ohshimo2009})。

\begin{center}
春季発生群: $BL_t = 143.96(1-e^{-0.15(t+0.44)})$\\
秋季発生群: $BL_t = 158.59(1-e^{-0.09(t+0.74)})$
\end{center}
ただし、$BL_t$は孵化後$t$ヶ月の被鱗体長(mm)である。
寿命は3年程度と考えられている。

\subsubsection{成熟・産卵}
カタクチイワシは、厳冬期を除いて周年にわたり産卵することが知られている。
若狭湾では体長8.5 cmで産卵することが報告されている\citep{Funamoto2004}。
鳥取県沿岸においては、体長11.9 cm以上であれば、ほとんどが産卵すると報告されている\citep{Shimura2008}。
これらの結果に従えば、春季発生群は翌年の産卵期にほぼ全て産卵することとなる。
そのため、本報告では満1歳から全個体が産卵に参加すると仮定した(図\ref{fig_mature_age})。

\subsubsection{被捕食関係}
カタクチイワシは、動物プランクトンのうち主にカイアシ類を餌料とする\citep{Tanaka2006}。
本種は多様な動物種の餌料となっており、
仔稚魚期にはマアジ・マサバなどの魚食性魚類や肉食性動物プランクトンに、
未成魚・成魚期には魚食性魚類の他に、クジラやイルカなどの海産ほ乳類や海鳥類などにも捕食される。


\subsection{漁業の状況}
\subsubsection{漁業の概要}
本系群は、日本海北区(石川県から新潟県)では主に定置網により漁獲され、
日本海西区(福井県から山口県)では主に大中型まき網・中型まき網・定置網などにより漁獲されている。
また、東シナ海区(福岡県から鹿児島県)では、主に中型まき網により漁獲される。
なお、シラスは主に熊本県や鹿児島県の沿岸域で漁獲されている。

\subsubsection{漁獲量の推移}
本系群の漁獲量は、漁業・養殖業生産統計年報の青森県~鹿児島県の合計値から、
東シナ海区に所属する漁船による太平洋海域における漁獲量(漁獲成績報告書による)
を差し引いた値とした(表\ref{table_catch}、図\ref{fig_catch_anchovy_shirasu})。
本系群の漁獲量は、1997年を除いて1996年から2000年までは100千トンを超えていたが、
その後2004年には61千トンにまで減少した。
漁獲量は2005年から2008年にかけて再び増加したが、2009年以降は減少傾向にあり、2015年は61千トンであった。
%
%
 海区別では、日本海北区の漁獲量は1995年に9千トンまで増加した後、1996年、2001年、2005年を除いて
 5千トン前後で変動していたが、2011年から2013年にかけて3千トンを下回った(表\ref{table_catch})。
 2015年の漁獲量は3千トンであった。
 %
 %
 %
日本海西区の漁獲量は、1991年から1998年にかけて70千トンまで増加したが、その後減少し、2001年以降は20千トン前後で推移した。
2015年は11千トンと少なかった(表\ref{table_catch})。
東シナ海区の漁獲量は、1990年から2000年(65千トン)まで増加傾向にあった。
その後は、2009年(26千トン)を除いて、2001から40~70千トンで推移しており、2015年は47千トンであった(表\ref{table_catch})。
%
%
対馬暖流域の沿岸域における仔魚(シラス)の漁獲量は、1977年以降1987年まで2千トンから6千トンの間で緩やかに増減したが、
それ以降10年間ほど6千トン前後の漁獲が維持された(表\ref{table_catch})。
漁獲量は1999年と2000年には10千トンを超えたが、2002年にかけて急減した。
漁獲量はその後、2005年前後に再び10千トン近くまで増加したが、2008年以降から減少傾向を示し、2015年には5千トンとなった。
%
%
韓国におけるカタクチイワシ漁獲量は、1995年以降20万トンを超えており、2000年以降は増減を繰り返している
(表\ref{table_catch};水産統計(韓国海洋水産部)、\url{http://www.fips.go.kr:7001/index.jsp}、2016年3月)。
2015年における漁獲量は21万トンであった。
韓国近海の漁場は韓国南岸および東岸である\citep{Korfish2000}。
%
%
中国の漁獲量は、日本・韓国よりも多く、1996年以降50万トン以上で維持されているが、
2003年に約111万トンとなって以降、2009年まで減少が続いた
(FAO Fishery and Aquaculture Statistics. Global capture production 1950--2014、
\url{http://www.fao.org/fishery/statistics/software/fishstatj/en}、2016年6月)。
中国の漁獲量は2009年以降増加しており(表\ref{table_catch})、
データが利用可能な直近年である2014年における値は93万トンであった。


\subsection{資源の状態}
\subsubsection{資源評価の方法}
シラスを含めた年別年齢別漁獲尾数に基づくコホート解析により資源量を推定した(補足資料\ref{appendix_stock_method})。
産卵量調査、計量魚探調査および新規加入量調査(ニューストンネット)などの結果は、
資源量を反映しているかの検討が不十分なため、コホート解析における資源量指標値としては用いず、資源動向などを判断するための参考値に留めた。

\subsubsection{資源量指標値の推移}
日本海と東シナ海における産卵量の推移を図\ref{fig_production}に示す。
産卵量は1998~2000年に多く、2001年には少なかったものの、2004年には合計10,084兆粒と1979年以降における最大値を示した。
その後、産卵量は増減を繰り返している。
2015年における産卵量の水準は日本海および東シナ海ともに中程度で、合計値は2,471兆粒であった。
%
%
夏季(8・9月)に九州北西岸で行われている、音響調査による現存量指標値\citep{Ohshimo2004}
および中層トロール調査のCPUE(漁獲尾数÷有効網数)を図\ref{fig_stock_index}に示す。
現存量指標値は増減を繰り返しながら推移しており、近年では2007年の134.0(相対値)が最も高かった。
現存量指標値はその後、急減し、2010~2012年は2.5~17.9と低水準で推移した。
2013年の現存量指標値は2007年の値の半分を超える程度まで回復し、2015年は108.8となった。
また、中層トロール調査のCPUEは、1990年代後半に比べると、2002年以降は低水準で変動している。
2015年のCPUEは67.4(kg/網)で、前年の値(12.3 kg/網)を大きく上回った。
%
%
九州北西岸で実施した調査において、ニューストンネットに入網したシラスのCPUEの推移を図\ref{fig_shirasuCPUE_kyushu}に示した。
6月に実施した調査におけるCPUEは、2003年(598尾/網)、2005年(815尾/網)、2009~2011年(475~928尾/網)に高い値を示したが、
2012年以降には299尾/網以下と低い水準にある。
%
8・9月の調査では、CPUEは2010年から2013年にかけて4~25尾/網と低い水準にあったが、2014年は214尾/網と大きく増加した。
しかし、2015年には67尾/網となり、前年を下回った。
その他主要魚種の採集個体数と、それに対応する有効曳網数は補足資料\ref{appendix_neuston}に示した。
4月に東シナ海で実施した調査において、ニューストンネットに入網したシラスのCPUEの推移を図\ref{fig_shirasuCPUE_ECS}に示した。
2003~2007年における値(385~765尾/網)に比べると、2008年~2010年は28~93尾/網と低い水準にあったが、
2011年以降増加傾向を示し、2015年には大幅に増加して1622尾/網となった。2015年は前年を下回り955尾/網であった。

\subsubsection{漁獲物の年齢組成}
本系群の年齢別漁獲尾数の推移を図\ref{fig_catch_age}と表\ref{table_cohort_data}に示した。
漁獲物のほとんどは0歳魚で、0歳魚の漁獲尾数には1977年以降、緩やかな増減が見られる。
0歳魚の漁獲尾数は、近年では1990年代後半と2000年代半ばに多かった。

\subsubsection{資源量と漁獲割合の推移}
コホート解析(補足資料\ref{appendix_stock_method})を用いて、本系群の資源尾数・漁獲係数(表\ref{table_stock_F})
及び資源量・親魚量・再生産成功率RPS(加入尾数÷親魚量)・漁獲割合(漁獲量÷資源量 $\times$100)(表\ref{table_stock_etc}、図\ref{fig_stock_percentage})を推定した。
1977年以降における資源量の最低値は1979年における74千トンであり、資源量はその後、増減を繰り返しながらも徐々に増加した。
資源量は1998年に306千トンの最大値を記録したが、2001年には130千トンにまで減少した。
資源量はその後、2007年まで再び増加傾向を示したが、2008年以降には減少傾向にある。
2015年の資源量は132千トンで、前年(120千トン)より増加したものの1987年以来の低水準であった。
漁獲割合は、1977年以降50\%前後で推移し、2015年の値は50\%だった。
 自然死亡係数($M$)を0.5、1.0(規定値)、1.5とした場合の資源量・親魚量・加入尾数の推定値を図\ref{fig_M_stock}に示した。
資源量は、$M$を0.5に仮定した場合には規定値の72\%となり、$M$を1.5に仮定した場合には144\%となった。

\subsubsection{Blimitの設定}
親魚量と加入尾数との関係を図\ref{fig_reproduction_Blimit}に示した。
親魚量と加入尾数は正の相関を示した。
RPSの上位10\%と加入尾数の上位10\%にそれぞれ相当する2直線の交点から、
資源回復の閾値となるBlimitを親魚量91千トン(2005年水準)とした。
2015年の親魚量は61千トンであり、Blimitを下回っている。
親魚量と加入量の経年変化を図\ref{fig_spawner_recruit}に、
RPSの経年変化を図\ref{fig_RPS}に示した。
RPSは増減を繰り返しながらも周期的な変化がみられる。
$F$(各年齢の$F$の平均値)とYPRおよび\%SPRの関係を図\ref{fig_F_SPR_YPR}に示した。
2015年の$F$(2.48)は$Fmed$(2.12)や$F$30\%SPR(1.29)、$Fmax$(0.91)、$F$0.1(0.60)よりも高い。

\subsubsection{資源の水準・動向}
Blimitである親魚量(91千トン)を資源水準の「低位」と「中位」の境界とした。
また親魚量の最小値を基準とした場合に、
親魚量の最大値までの増分の上位1/3と2/3の境界(155千トン)を「高位」と「中位」の境界とした。
なお、同様の方法において下位1/3にあたる親魚量は100千トンで、これはBlimitに比較的近似している。
2015年の親魚量(61千トン)がBlimitを下回っていることから、資源の水準を低位と判断した。
動向は、過去5年(2011年~2015年)の資源量と親魚量の推移から横ばいと判断した。

\subsubsection{資源と漁獲の関係}
資源量と漁獲係数($F$)との間に明瞭な関係は見られなかった(図\ref{fig_stock_F})。


\subsection{\thisyrad 年ABCの算定}
\subsubsection{資源評価のまとめ}
コホート解析によると2015年の親魚量は61千トンであり、
これは再生産関係(図\ref{fig_spawner_recruit})から求められるBlimit(親魚量91千トン)を下回っている。
資源量と親魚量はともに2011年以降、横ばい傾向にある。
以上を根拠に、資源水準を低位、動向を横ばいと判断した。

\subsubsection{ABCの算定}
本系群では、資源量および再生産関係が明らかとなっており、また親魚量がBlimitを下回っているため、ABC算定ルール1-1)-(2)を用い、
5年後(2021年)に親魚量をBlimitまで回復させる$F$($Frec5yr$)を管理基準値として、2017年ABCを算出した。
ABC算定のための式は次の通りである。
\begin{eqnarray*}
Flimit &=& Frec5yc\\
Ftarget &=& \alpha Flimit
\end{eqnarray*}
Flimitは、5年後(2021年)に親魚量がBlimitまで回復する$F$($Frec5yr$)とし、$\alpha$は基準値の0.8とした。
2016年の$F$は$Fcurrent$($F2015$)とし、
2016年以降の再生産成功率は、直近年を除く過去10年間(2005~2014年)の中央値(777尾/kg)で推移すると仮定した。
また、加入尾数の上限を過去10年間(2006~2015年)の最大値(1,293億尾)と仮定した。
算出したABCは、以下の通りである。
なお、ABCはシラスの漁獲量を含む。

★表を入れる★
★表を入れる★
★表を入れる★

Targetは、資源変動の可能性やデータ誤差に起因する評価の不確実性を考慮し、より安定的な資源の増大が期待される漁獲量である。
Limitは、管理基準の下で許容される最大レベルの漁獲量である。$Ftarget = \alpha Flimit$とし、係数$\alpha$には標準値0.8を用いた。
漁獲割合は、漁獲量÷資源量 $\times100$である。$F$は各年齢の平均である。

\subsubsection{ABCの評価}
$Frec5yr$、$0.8Frec5yr$および$Fcurrent$のもとでの資源量、漁獲量、親魚量の変化を図\ref{}に示した。
さらに、$Fcurrent$に様々な係数を乗じた際の資源量と漁獲量の変化を以下の表に示す。
資源量は、$Fcurrent$においては継続して減少するが、$F$を低下させた場合には2017年以降に増加するため、これに伴う漁獲量の増加が期待される。

★表を入れる★
★表を入れる★
★表を入れる★
★表を入れる★

\subsubsection{ABCの再評価}
2015年(2016年再評価)では、2014年の漁獲量および2015年における年齢別体重を更新した。
また、再生産成功率を本年度評価と同一と仮定し、2019年における親魚量がBlimitへ回復する$F$を求めた。
2016年(2016年再評価)では、再評価時の最近年の資源量推定結果を用いて、2020年における親魚量がBlimitへ回復する$F$を求めた。
資源量推定値は昨年度評価時の値を上回り、やや高めの$F$でも資源回復が可能となったため、2016年のABCはやや多く見積もられた。
この主な要因は、2015年の0歳魚の体重および漁獲尾数が昨年度の予測より大きく、2015年の年齢別体重に基づく将来の親魚量がより多く見積もられたためである。
平成27年度まで本系群の資源評価報告書では、$Frec5yr$を$Frec$と表記していた。


\subsection{ABC以外の管理方策の提言}
本種は寿命が短く、漁獲物の大半は0歳魚である。
親魚量と加入尾数には正の相関が見られることから、資源を安定して利用するためには、親魚量を一定以上に保つことが有効である。
そのため、加入が少ないと判断された場合には、0歳魚を獲り控えることが効果的と考えられる。


\bibliographystyle{mynatbib}
\bibliography{output/TW_Sardinops-melanostictus_references}%         引用文献がここに出る。
\clearpage%                         改ページ
\TwoOfSixFigs
{fig/bunpu.pdf}{カタクチイワシ対馬暖流系群の分布域}{fig_bunpu}
{fig/seichou.pdf}{カタクチイワシの成長様式\newline◯: 春季発生群観測値、■: 秋季発生群観測値、△: 年齢別体重、実線: 春季発生群成長式、破線: 秋季発生群成長式。}{fig_seichou}

\TwoOfSixFigs
{fig/mature_age.pdf}{年齢別成熟率}{fig_mature_age}
{fig/catch_anchovy_shirasu.pdf}{カタクチイワシとシラスの漁獲量}{fig_catch_anchovy_shirasu}

\TwoOfSixFigs
{fig/production.pdf}{産卵量の経年変化}{fig_production}
{fig/stock_index.pdf}{現存量指標値}{fig_stock_index}

\TwoOfSixFigs
{fig/shirasuCPUE_kyushu.pdf}{九州北西岸におけるシラス調査CPUE}{fig_shirasuCPUE_kyushu}
{fig/shirasuCPUE_ECS.pdf}{東シナ海におけるシラス調査CPUE}{fig_shirasuCPUE_ECS}

\TwoOfSixFigs
{fig/catch_age.pdf}{年齢別漁獲尾数}{fig_catch_age}
{fig/stock_percentage.pdf}{推定された資源量と漁獲割合}{fig_stock_percentage}

\TwoOfSixFigs
{fig/M_stock.pdf}{自然死亡係数($M$)の変化に伴う資源量、親魚量および加入尾数の変化}{fig_M_stock}
{fig/reproduction_Blimit.pdf}{再生産関係とBlimit(Bblimit)の設定}{fig_reproduction_Blimit}

\TwoOfSixFigs
{fig/spawner_recruit.pdf}{親魚量と加入尾数の経年変化}{fig_spawner_recruit}
{fig/RPS.pdf}{RPSの経年変化}{fig_RPS}

\TwoOfSixFigs
{fig/F_SPR_YPR.pdf}{漁獲係数($F$)と\%SPR(実線)およびYPR(破線)との関係}{fig_F_SPR_YPR}
{fig/stock_F.pdf}{資源量と漁獲係数($F$)との関係}{fig_stock_F}

\OneOfSixFigs
{fig/F_SPR_YPR.pdf}{漁獲係数($F$)と\%SPR(実線)およびYPR(破線)との関係}{fig_F_SPR_YPR2}


\TwoOfEightFigs
{fig/maaji_bunpu.pdf}{a}{a}
{fig/maaji_gyojou.pdf}{a}{b}
\TwoOfEightFigs
{fig/maaji_daichu.pdf}{a}{c}
{fig/maaji_gyokaku.pdf}{a}{d}
\TwoOfEightFigs
{fig/maaji_index.pdf}{a}{e}
{fig/maaji_nenrei.pdf}{a}{f}
\TwoOfEightFigs
{fig/maaji_nenreibetu.pdf}{a}{g}
{fig/maaji_seijuku.pdf}{a}{h}
%                   図を読み込み
%\input{_tables}
\setcounter{chapter}{0}

%% このファイルは編集は不要。
\chapter{\thisyrjp(\thisyrad)年度カタクチイワシ対馬暖流系群の資源評価}% 今年の年数を「jp(和暦)」と「ad(西暦)」で呼び出し。
%
\担当機関等
{西海区水産研究所}%担当水研
{安田十也、林 晃、黒田啓行、\CID{8705}橋素光}%担当者
{日本海区水産研究所、青森県産業技術センター水産総合研究所、秋田県水産振興センター、山形県水産試験場、新潟県水産海洋研究所、富山県農林水産総合技術センター水産研究所、石川県水産総合センター、福井県水産試験場、京都府農林水産技術センター海洋センター、兵庫県立農林水産技術総合センター但馬水産技術センター、鳥取県水産試験場、島根県水産技術センター、山口県水産研究センター、福岡県水産海洋技術センター、佐賀県玄海水産振興センター、長崎県総合水産試験場、熊本県水産研究センター、鹿児島県水産技術開発センター}%関連機関

\要約
本系群の資源量について、コホート解析により計算した。
資源量は1995年から2000年まで200千トン以上であったが、2001年に130千トンへ減少した。
2004年以降資源量は増加し、2007年には247千トンとなったが、それ以降減少傾向を示した。
2015年における資源量は132千トンと推定され、前年(120千トン)より増加した。
過去の資源量と親魚量から資源水準は低位、過去5年間(2011~2015年)の資源量の推移から動向は横ばいと判断した。
再生産関係から、Blimitを2005年水準の親魚量91千トンとした。
2015年の親魚量(61千トン)はBlimitを下回っている。
5年後に親魚量をBlimitまで回復させる$F$($Frec5yr$)を管理基準値として、2017年ABCを算出した。%$マークではさむと数式モードがオンに
ただし、
本報告での
ABCは仔魚
(シラス)
を
含む
日本の
漁獲に対する値である。%単発の改行は無視される。

\begin{center}
\begin{tabularx}{14.1cm}{cccccc}
\toprule
\multirow{2}{*}{管理基準}	& {Target/Limit} 	& {$F$} 	& {漁獲割合(\%)} 	& {\shortstack{\\\thisyrad 年ABC\\(千トン)}} 	& Blimit = 91\newline(千トン)\tabularnewline \cline{6-6}
						& 					& 		& 					& 						& 親魚量5年後(千トン)\tabularnewline
\hline
\multirow{2}{*}{$Frec5yr$}& Target 			& 1.55 	& 44 				& 47 					& 222				\tabularnewline \cline{2-6}
						& Limit				& 1.94	& 48				& 51					& ~91				\tabularnewline
\bottomrule
\end{tabularx}
\end{center}
%表を読み込み

Targetは、資源変動の可能性やデータ誤差に起因する評価の不確実性を考慮し、
より安定的な資源の増大が期待される漁獲量である。
Limitは、管理基準の下で許容される最大レベルの漁獲量である。
$Ftarget = \alpha Flimit$とし、係数$\alpha$には標準値0.8を用いた。
漁獲割合は、漁獲量÷資源量とした。$F$値は各年齢の平均とした。
2015年の親魚量は61千トン。
ABCはシラスの漁獲量を含む。
$Frec5yr$は5年後に親魚量をBlimitまで回復させる$F$。
\過去五年間の資源量等{2012 & 106 & 56 & 55 & 2.21 & 51}{2013 & 101 & 71 & 52 & 2.10 & 52}{2014 & 120 & 78 & 64 & 3.14 & 54}{2015 & 132 & 61 & 66 & 2.48 & 50}{2016 & 131 & 67 & -- & -- & --}


ただし、$F$は各年齢の単純平均。
シラスの漁獲量を含む。
\thisyrad 年の資源量・親魚量は加入尾数を仮定した値。%今年を呼び出し。
%        要約ファイルを読み込み。以下同じなので略。
\subsection{まえがき}
我が国周辺に分布するカタクチイワシは、太平洋系群、瀬戸内海系群および対馬暖流系群から構成される。
本種の漁獲量は、マイワシとは対照的に1990年代に増加した。対馬暖流域においても、1990年代後半にかけて漁獲量が増加したが、
2001年に急減し、その後は増減を繰り返している。しかし、本種の漁獲量の変動幅はマイワシほど大きくない。
これは、マイワシと比較して親魚になるまでの期間が短いことや、ほぼ周年にわたり産卵を行うことなどが要因と考えられる。

東シナ海や日本海に分布するカタクチイワシは、韓国や中国によっても漁獲されているが、これらの主な分布域は韓国と中国の沿岸域であるため、
対馬暖流系群とはみなさず、本資源評価では考慮しなかった。

\subsection{分布・回遊} 
マアジ太平洋系群の分布域を図1に、主な漁場形成の模式図を図2に示した。
日本近海のうち太平洋および隣接海域に分布するマアジには、東シナ海を主産卵場とする群と本州中部以南で産卵する地先群があると考えられている。
太平洋沿岸の中部以東の海域では加入時期の異なる群が見られ、2~4月に東シナ海で生まれたものと5月以降に太平洋沿岸域で生まれたものが分布すると考えられている(木幡 1972)。
また、東シナ海からの加入群(横田・三田 1958)の多寡が資源水準を左右するとも考えられている(古藤 1990)。
我が国近海のマアジ資源は東シナ海に本系群と対馬暖流系群共通の産卵場があると考えられるため、両系群あわせて評価することも想定されるが、本系群の親魚が東シナ海に産卵回遊する情報もないため、結論は得られていない。

\subsubsection{年齢・成長}
1年で尾叉長18cm、2年で24cm程度に成長する(図3)。
寿命は5歳前後と考えられるが、4歳魚以上の漁獲は少ない。

\subsubsection{成熟・産卵}
産卵期は南部ほど早く、豊後水道、紀伊水道外域などでは冬から初夏であり(阪本ほか 1986、薬師寺 2001、阪地2001)、相模湾では春から初夏(木幡 1972、澤田 1974)である。
1歳で50%、2歳以上で100%が成熟する(図4)。

\subsubsection{被捕食関係}
仔稚魚は成長するにつれて大型の動物プランクトンを摂餌し、幼魚以降では魚食性が強くなる(三谷ほか 2001)。
本種は大型の魚類等により捕食される(三谷ほか 2001)。

\subsection{漁業の状況}
\input{output/gyogyou_gaiyou}
\input{output/gyokakuryou_suii}

\subsection{資源の状態}
\input{output/shigenhyouka_houhou}
\input{output/shigenryou_shihyouchi}
\input{output/gyokakubutsu_nenreisosei}
\input{output/shigenryou_gyokakuwariai}
\input{output/blimit_settei}
\input{output/shigen_suijun_doukou}
\input{output/shigen_gyokaku}

\subsection{\thisyrad 年ABCの算定}
\input{output/shigenhyouka_matome}
\input{output/abc_santei}
\input{output/abc_hyouka}
\input{output/abc_saihyouka}

\subsection{ABC以外の管理方策の提言}
本種は寿命が短く、漁獲物の大半は0歳魚である。
親魚量と加入尾数には正の相関が見られることから、資源を安定して利用するためには、親魚量を一定以上に保つことが有効である。
そのため、加入が少ないと判断された場合には、0歳魚を獲り控えることが効果的と考えられる。


\bibliographystyle{mynatbib}%       文献の引用スタイルを指定。ほんとは総元締めファイルに書きたいが、魚種ごとに文献リストが必要なので仕方ない。
\bibliography{output/TW_Engraulis-japonicus_references}%         引用文献がここに出る。
\clearpage%                         改ページ
\TwoOfSixFigs
{fig/bunpu.pdf}{カタクチイワシ対馬暖流系群の分布域}{fig_bunpu}
{fig/seichou.pdf}{カタクチイワシの成長様式\newline◯: 春季発生群観測値、■: 秋季発生群観測値、△: 年齢別体重、実線: 春季発生群成長式、破線: 秋季発生群成長式。}{fig_seichou}

\TwoOfSixFigs
{fig/mature_age.pdf}{年齢別成熟率}{fig_mature_age}
{fig/catch_anchovy_shirasu.pdf}{カタクチイワシとシラスの漁獲量}{fig_catch_anchovy_shirasu}

\TwoOfSixFigs
{fig/production.pdf}{産卵量の経年変化}{fig_production}
{fig/stock_index.pdf}{現存量指標値}{fig_stock_index}

\TwoOfSixFigs
{fig/shirasuCPUE_kyushu.pdf}{九州北西岸におけるシラス調査CPUE}{fig_shirasuCPUE_kyushu}
{fig/shirasuCPUE_ECS.pdf}{東シナ海におけるシラス調査CPUE}{fig_shirasuCPUE_ECS}

\TwoOfSixFigs
{fig/catch_age.pdf}{年齢別漁獲尾数}{fig_catch_age}
{fig/stock_percentage.pdf}{推定された資源量と漁獲割合}{fig_stock_percentage}

\TwoOfSixFigs
{fig/M_stock.pdf}{自然死亡係数($M$)の変化に伴う資源量、親魚量および加入尾数の変化}{fig_M_stock}
{fig/reproduction_Blimit.pdf}{再生産関係とBlimit(Bblimit)の設定}{fig_reproduction_Blimit}

\TwoOfSixFigs
{fig/spawner_recruit.pdf}{親魚量と加入尾数の経年変化}{fig_spawner_recruit}
{fig/RPS.pdf}{RPSの経年変化}{fig_RPS}

\TwoOfSixFigs
{fig/F_SPR_YPR.pdf}{漁獲係数($F$)と\%SPR(実線)およびYPR(破線)との関係}{fig_F_SPR_YPR}
{fig/stock_F.pdf}{資源量と漁獲係数($F$)との関係}{fig_stock_F}

\OneOfSixFigs
{fig/F_SPR_YPR.pdf}{漁獲係数($F$)と\%SPR(実線)およびYPR(破線)との関係}{fig_F_SPR_YPR2}


\TwoOfEightFigs
{fig/maaji_bunpu.pdf}{a}{a}
{fig/maaji_gyojou.pdf}{a}{b}
\TwoOfEightFigs
{fig/maaji_daichu.pdf}{a}{c}
{fig/maaji_gyokaku.pdf}{a}{d}
\TwoOfEightFigs
{fig/maaji_index.pdf}{a}{e}
{fig/maaji_nenrei.pdf}{a}{f}
\TwoOfEightFigs
{fig/maaji_nenreibetu.pdf}{a}{g}
{fig/maaji_seijuku.pdf}{a}{h}
%                   図を読み込み
%\input{_tables}
\setcounter{chapter}{0}

%\part*{第2分冊}
%\include{_PF_Engraulis-japonicus}
%\include{_SI_Engraulis-japonicus}
%% このファイルは編集は不要。
\chapter{\thisyrjp(\thisyrad)年度カタクチイワシ対馬暖流系群の資源評価}% 今年の年数を「jp(和暦)」と「ad(西暦)」で呼び出し。
%
\担当機関等
{西海区水産研究所}%担当水研
{安田十也、林 晃、黒田啓行、\CID{8705}橋素光}%担当者
{日本海区水産研究所、青森県産業技術センター水産総合研究所、秋田県水産振興センター、山形県水産試験場、新潟県水産海洋研究所、富山県農林水産総合技術センター水産研究所、石川県水産総合センター、福井県水産試験場、京都府農林水産技術センター海洋センター、兵庫県立農林水産技術総合センター但馬水産技術センター、鳥取県水産試験場、島根県水産技術センター、山口県水産研究センター、福岡県水産海洋技術センター、佐賀県玄海水産振興センター、長崎県総合水産試験場、熊本県水産研究センター、鹿児島県水産技術開発センター}%関連機関

\要約
本系群の資源量について、コホート解析により計算した。
資源量は1995年から2000年まで200千トン以上であったが、2001年に130千トンへ減少した。
2004年以降資源量は増加し、2007年には247千トンとなったが、それ以降減少傾向を示した。
2015年における資源量は132千トンと推定され、前年(120千トン)より増加した。
過去の資源量と親魚量から資源水準は低位、過去5年間(2011~2015年)の資源量の推移から動向は横ばいと判断した。
再生産関係から、Blimitを2005年水準の親魚量91千トンとした。
2015年の親魚量(61千トン)はBlimitを下回っている。
5年後に親魚量をBlimitまで回復させる$F$($Frec5yr$)を管理基準値として、2017年ABCを算出した。%$マークではさむと数式モードがオンに
ただし、
本報告での
ABCは仔魚
(シラス)
を
含む
日本の
漁獲に対する値である。%単発の改行は無視される。

\begin{center}
\begin{tabularx}{14.1cm}{cccccc}
\toprule
\multirow{2}{*}{管理基準}	& {Target/Limit} 	& {$F$} 	& {漁獲割合(\%)} 	& {\shortstack{\\\thisyrad 年ABC\\(千トン)}} 	& Blimit = 91\newline(千トン)\tabularnewline \cline{6-6}
						& 					& 		& 					& 						& 親魚量5年後(千トン)\tabularnewline
\hline
\multirow{2}{*}{$Frec5yr$}& Target 			& 1.55 	& 44 				& 47 					& 222				\tabularnewline \cline{2-6}
						& Limit				& 1.94	& 48				& 51					& ~91				\tabularnewline
\bottomrule
\end{tabularx}
\end{center}
%表を読み込み

Targetは、資源変動の可能性やデータ誤差に起因する評価の不確実性を考慮し、
より安定的な資源の増大が期待される漁獲量である。
Limitは、管理基準の下で許容される最大レベルの漁獲量である。
$Ftarget = \alpha Flimit$とし、係数$\alpha$には標準値0.8を用いた。
漁獲割合は、漁獲量÷資源量とした。$F$値は各年齢の平均とした。
2015年の親魚量は61千トン。
ABCはシラスの漁獲量を含む。
$Frec5yr$は5年後に親魚量をBlimitまで回復させる$F$。
\過去五年間の資源量等{2012 & 106 & 56 & 55 & 2.21 & 51}{2013 & 101 & 71 & 52 & 2.10 & 52}{2014 & 120 & 78 & 64 & 3.14 & 54}{2015 & 132 & 61 & 66 & 2.48 & 50}{2016 & 131 & 67 & -- & -- & --}


ただし、$F$は各年齢の単純平均。
シラスの漁獲量を含む。
\thisyrad 年の資源量・親魚量は加入尾数を仮定した値。%今年を呼び出し。
%        要約ファイルを読み込み。以下同じなので略。
\subsection{まえがき}
我が国周辺に分布するカタクチイワシは、太平洋系群、瀬戸内海系群および対馬暖流系群から構成される。
本種の漁獲量は、マイワシとは対照的に1990年代に増加した。対馬暖流域においても、1990年代後半にかけて漁獲量が増加したが、
2001年に急減し、その後は増減を繰り返している。しかし、本種の漁獲量の変動幅はマイワシほど大きくない。
これは、マイワシと比較して親魚になるまでの期間が短いことや、ほぼ周年にわたり産卵を行うことなどが要因と考えられる。

東シナ海や日本海に分布するカタクチイワシは、韓国や中国によっても漁獲されているが、これらの主な分布域は韓国と中国の沿岸域であるため、
対馬暖流系群とはみなさず、本資源評価では考慮しなかった。

\subsection{分布・回遊} 
マアジ太平洋系群の分布域を図1に、主な漁場形成の模式図を図2に示した。
日本近海のうち太平洋および隣接海域に分布するマアジには、東シナ海を主産卵場とする群と本州中部以南で産卵する地先群があると考えられている。
太平洋沿岸の中部以東の海域では加入時期の異なる群が見られ、2~4月に東シナ海で生まれたものと5月以降に太平洋沿岸域で生まれたものが分布すると考えられている(木幡 1972)。
また、東シナ海からの加入群(横田・三田 1958)の多寡が資源水準を左右するとも考えられている(古藤 1990)。
我が国近海のマアジ資源は東シナ海に本系群と対馬暖流系群共通の産卵場があると考えられるため、両系群あわせて評価することも想定されるが、本系群の親魚が東シナ海に産卵回遊する情報もないため、結論は得られていない。

\subsubsection{年齢・成長}
1年で尾叉長18cm、2年で24cm程度に成長する(図3)。
寿命は5歳前後と考えられるが、4歳魚以上の漁獲は少ない。

\subsubsection{成熟・産卵}
産卵期は南部ほど早く、豊後水道、紀伊水道外域などでは冬から初夏であり(阪本ほか 1986、薬師寺 2001、阪地2001)、相模湾では春から初夏(木幡 1972、澤田 1974)である。
1歳で50%、2歳以上で100%が成熟する(図4)。

\subsubsection{被捕食関係}
仔稚魚は成長するにつれて大型の動物プランクトンを摂餌し、幼魚以降では魚食性が強くなる(三谷ほか 2001)。
本種は大型の魚類等により捕食される(三谷ほか 2001)。

\subsection{漁業の状況}
\input{output/gyogyou_gaiyou}
\input{output/gyokakuryou_suii}

\subsection{資源の状態}
\input{output/shigenhyouka_houhou}
\input{output/shigenryou_shihyouchi}
\input{output/gyokakubutsu_nenreisosei}
\input{output/shigenryou_gyokakuwariai}
\input{output/blimit_settei}
\input{output/shigen_suijun_doukou}
\input{output/shigen_gyokaku}

\subsection{\thisyrad 年ABCの算定}
\input{output/shigenhyouka_matome}
\input{output/abc_santei}
\input{output/abc_hyouka}
\input{output/abc_saihyouka}

\subsection{ABC以外の管理方策の提言}
本種は寿命が短く、漁獲物の大半は0歳魚である。
親魚量と加入尾数には正の相関が見られることから、資源を安定して利用するためには、親魚量を一定以上に保つことが有効である。
そのため、加入が少ないと判断された場合には、0歳魚を獲り控えることが効果的と考えられる。


\bibliographystyle{mynatbib}%       文献の引用スタイルを指定。ほんとは総元締めファイルに書きたいが、魚種ごとに文献リストが必要なので仕方ない。
\bibliography{output/TW_Engraulis-japonicus_references}%         引用文献がここに出る。
\clearpage%                         改ページ
\TwoOfSixFigs
{fig/bunpu.pdf}{カタクチイワシ対馬暖流系群の分布域}{fig_bunpu}
{fig/seichou.pdf}{カタクチイワシの成長様式\newline◯: 春季発生群観測値、■: 秋季発生群観測値、△: 年齢別体重、実線: 春季発生群成長式、破線: 秋季発生群成長式。}{fig_seichou}

\TwoOfSixFigs
{fig/mature_age.pdf}{年齢別成熟率}{fig_mature_age}
{fig/catch_anchovy_shirasu.pdf}{カタクチイワシとシラスの漁獲量}{fig_catch_anchovy_shirasu}

\TwoOfSixFigs
{fig/production.pdf}{産卵量の経年変化}{fig_production}
{fig/stock_index.pdf}{現存量指標値}{fig_stock_index}

\TwoOfSixFigs
{fig/shirasuCPUE_kyushu.pdf}{九州北西岸におけるシラス調査CPUE}{fig_shirasuCPUE_kyushu}
{fig/shirasuCPUE_ECS.pdf}{東シナ海におけるシラス調査CPUE}{fig_shirasuCPUE_ECS}

\TwoOfSixFigs
{fig/catch_age.pdf}{年齢別漁獲尾数}{fig_catch_age}
{fig/stock_percentage.pdf}{推定された資源量と漁獲割合}{fig_stock_percentage}

\TwoOfSixFigs
{fig/M_stock.pdf}{自然死亡係数($M$)の変化に伴う資源量、親魚量および加入尾数の変化}{fig_M_stock}
{fig/reproduction_Blimit.pdf}{再生産関係とBlimit(Bblimit)の設定}{fig_reproduction_Blimit}

\TwoOfSixFigs
{fig/spawner_recruit.pdf}{親魚量と加入尾数の経年変化}{fig_spawner_recruit}
{fig/RPS.pdf}{RPSの経年変化}{fig_RPS}

\TwoOfSixFigs
{fig/F_SPR_YPR.pdf}{漁獲係数($F$)と\%SPR(実線)およびYPR(破線)との関係}{fig_F_SPR_YPR}
{fig/stock_F.pdf}{資源量と漁獲係数($F$)との関係}{fig_stock_F}

\OneOfSixFigs
{fig/F_SPR_YPR.pdf}{漁獲係数($F$)と\%SPR(実線)およびYPR(破線)との関係}{fig_F_SPR_YPR2}


\TwoOfEightFigs
{fig/maaji_bunpu.pdf}{a}{a}
{fig/maaji_gyojou.pdf}{a}{b}
\TwoOfEightFigs
{fig/maaji_daichu.pdf}{a}{c}
{fig/maaji_gyokaku.pdf}{a}{d}
\TwoOfEightFigs
{fig/maaji_index.pdf}{a}{e}
{fig/maaji_nenrei.pdf}{a}{f}
\TwoOfEightFigs
{fig/maaji_nenreibetu.pdf}{a}{g}
{fig/maaji_seijuku.pdf}{a}{h}
%                   図を読み込み
%\input{_tables}
\setcounter{chapter}{0}

\subfile{fish1}
\subfile{fish2}


%\part*{第2分冊}
%% このファイルは編集は不要。
\chapter{\thisyrjp(\thisyrad)年度カタクチイワシ対馬暖流系群の資源評価}% 今年の年数を「jp(和暦)」と「ad(西暦)」で呼び出し。
%
\担当機関等
{西海区水産研究所}%担当水研
{安田十也、林 晃、黒田啓行、\CID{8705}橋素光}%担当者
{日本海区水産研究所、青森県産業技術センター水産総合研究所、秋田県水産振興センター、山形県水産試験場、新潟県水産海洋研究所、富山県農林水産総合技術センター水産研究所、石川県水産総合センター、福井県水産試験場、京都府農林水産技術センター海洋センター、兵庫県立農林水産技術総合センター但馬水産技術センター、鳥取県水産試験場、島根県水産技術センター、山口県水産研究センター、福岡県水産海洋技術センター、佐賀県玄海水産振興センター、長崎県総合水産試験場、熊本県水産研究センター、鹿児島県水産技術開発センター}%関連機関

\要約
本系群の資源量について、コホート解析により計算した。
資源量は1995年から2000年まで200千トン以上であったが、2001年に130千トンへ減少した。
2004年以降資源量は増加し、2007年には247千トンとなったが、それ以降減少傾向を示した。
2015年における資源量は132千トンと推定され、前年(120千トン)より増加した。
過去の資源量と親魚量から資源水準は低位、過去5年間(2011~2015年)の資源量の推移から動向は横ばいと判断した。
再生産関係から、Blimitを2005年水準の親魚量91千トンとした。
2015年の親魚量(61千トン)はBlimitを下回っている。
5年後に親魚量をBlimitまで回復させる$F$($Frec5yr$)を管理基準値として、2017年ABCを算出した。%$マークではさむと数式モードがオンに
ただし、
本報告での
ABCは仔魚
(シラス)
を
含む
日本の
漁獲に対する値である。%単発の改行は無視される。

\begin{center}
\begin{tabularx}{14.1cm}{cccccc}
\toprule
\multirow{2}{*}{管理基準}	& {Target/Limit} 	& {$F$} 	& {漁獲割合(\%)} 	& {\shortstack{\\\thisyrad 年ABC\\(千トン)}} 	& Blimit = 91\newline(千トン)\tabularnewline \cline{6-6}
						& 					& 		& 					& 						& 親魚量5年後(千トン)\tabularnewline
\hline
\multirow{2}{*}{$Frec5yr$}& Target 			& 1.55 	& 44 				& 47 					& 222				\tabularnewline \cline{2-6}
						& Limit				& 1.94	& 48				& 51					& ~91				\tabularnewline
\bottomrule
\end{tabularx}
\end{center}
%表を読み込み

Targetは、資源変動の可能性やデータ誤差に起因する評価の不確実性を考慮し、
より安定的な資源の増大が期待される漁獲量である。
Limitは、管理基準の下で許容される最大レベルの漁獲量である。
$Ftarget = \alpha Flimit$とし、係数$\alpha$には標準値0.8を用いた。
漁獲割合は、漁獲量÷資源量とした。$F$値は各年齢の平均とした。
2015年の親魚量は61千トン。
ABCはシラスの漁獲量を含む。
$Frec5yr$は5年後に親魚量をBlimitまで回復させる$F$。
\過去五年間の資源量等{2012 & 106 & 56 & 55 & 2.21 & 51}{2013 & 101 & 71 & 52 & 2.10 & 52}{2014 & 120 & 78 & 64 & 3.14 & 54}{2015 & 132 & 61 & 66 & 2.48 & 50}{2016 & 131 & 67 & -- & -- & --}


ただし、$F$は各年齢の単純平均。
シラスの漁獲量を含む。
\thisyrad 年の資源量・親魚量は加入尾数を仮定した値。%今年を呼び出し。
%        要約ファイルを読み込み。以下同じなので略。
\subsection{まえがき}
我が国周辺に分布するカタクチイワシは、太平洋系群、瀬戸内海系群および対馬暖流系群から構成される。
本種の漁獲量は、マイワシとは対照的に1990年代に増加した。対馬暖流域においても、1990年代後半にかけて漁獲量が増加したが、
2001年に急減し、その後は増減を繰り返している。しかし、本種の漁獲量の変動幅はマイワシほど大きくない。
これは、マイワシと比較して親魚になるまでの期間が短いことや、ほぼ周年にわたり産卵を行うことなどが要因と考えられる。

東シナ海や日本海に分布するカタクチイワシは、韓国や中国によっても漁獲されているが、これらの主な分布域は韓国と中国の沿岸域であるため、
対馬暖流系群とはみなさず、本資源評価では考慮しなかった。

\subsection{分布・回遊} 
マアジ太平洋系群の分布域を図1に、主な漁場形成の模式図を図2に示した。
日本近海のうち太平洋および隣接海域に分布するマアジには、東シナ海を主産卵場とする群と本州中部以南で産卵する地先群があると考えられている。
太平洋沿岸の中部以東の海域では加入時期の異なる群が見られ、2~4月に東シナ海で生まれたものと5月以降に太平洋沿岸域で生まれたものが分布すると考えられている(木幡 1972)。
また、東シナ海からの加入群(横田・三田 1958)の多寡が資源水準を左右するとも考えられている(古藤 1990)。
我が国近海のマアジ資源は東シナ海に本系群と対馬暖流系群共通の産卵場があると考えられるため、両系群あわせて評価することも想定されるが、本系群の親魚が東シナ海に産卵回遊する情報もないため、結論は得られていない。

\subsubsection{年齢・成長}
1年で尾叉長18cm、2年で24cm程度に成長する(図3)。
寿命は5歳前後と考えられるが、4歳魚以上の漁獲は少ない。

\subsubsection{成熟・産卵}
産卵期は南部ほど早く、豊後水道、紀伊水道外域などでは冬から初夏であり(阪本ほか 1986、薬師寺 2001、阪地2001)、相模湾では春から初夏(木幡 1972、澤田 1974)である。
1歳で50%、2歳以上で100%が成熟する(図4)。

\subsubsection{被捕食関係}
仔稚魚は成長するにつれて大型の動物プランクトンを摂餌し、幼魚以降では魚食性が強くなる(三谷ほか 2001)。
本種は大型の魚類等により捕食される(三谷ほか 2001)。

\subsection{漁業の状況}
\input{output/gyogyou_gaiyou}
\input{output/gyokakuryou_suii}

\subsection{資源の状態}
\input{output/shigenhyouka_houhou}
\input{output/shigenryou_shihyouchi}
\input{output/gyokakubutsu_nenreisosei}
\input{output/shigenryou_gyokakuwariai}
\input{output/blimit_settei}
\input{output/shigen_suijun_doukou}
\input{output/shigen_gyokaku}

\subsection{\thisyrad 年ABCの算定}
\input{output/shigenhyouka_matome}
\input{output/abc_santei}
\input{output/abc_hyouka}
\input{output/abc_saihyouka}

\subsection{ABC以外の管理方策の提言}
本種は寿命が短く、漁獲物の大半は0歳魚である。
親魚量と加入尾数には正の相関が見られることから、資源を安定して利用するためには、親魚量を一定以上に保つことが有効である。
そのため、加入が少ないと判断された場合には、0歳魚を獲り控えることが効果的と考えられる。


\bibliographystyle{mynatbib}%       文献の引用スタイルを指定。ほんとは総元締めファイルに書きたいが、魚種ごとに文献リストが必要なので仕方ない。
\bibliography{output/TW_Engraulis-japonicus_references}%         引用文献がここに出る。
\clearpage%                         改ページ
\TwoOfSixFigs
{fig/bunpu.pdf}{カタクチイワシ対馬暖流系群の分布域}{fig_bunpu}
{fig/seichou.pdf}{カタクチイワシの成長様式\newline◯: 春季発生群観測値、■: 秋季発生群観測値、△: 年齢別体重、実線: 春季発生群成長式、破線: 秋季発生群成長式。}{fig_seichou}

\TwoOfSixFigs
{fig/mature_age.pdf}{年齢別成熟率}{fig_mature_age}
{fig/catch_anchovy_shirasu.pdf}{カタクチイワシとシラスの漁獲量}{fig_catch_anchovy_shirasu}

\TwoOfSixFigs
{fig/production.pdf}{産卵量の経年変化}{fig_production}
{fig/stock_index.pdf}{現存量指標値}{fig_stock_index}

\TwoOfSixFigs
{fig/shirasuCPUE_kyushu.pdf}{九州北西岸におけるシラス調査CPUE}{fig_shirasuCPUE_kyushu}
{fig/shirasuCPUE_ECS.pdf}{東シナ海におけるシラス調査CPUE}{fig_shirasuCPUE_ECS}

\TwoOfSixFigs
{fig/catch_age.pdf}{年齢別漁獲尾数}{fig_catch_age}
{fig/stock_percentage.pdf}{推定された資源量と漁獲割合}{fig_stock_percentage}

\TwoOfSixFigs
{fig/M_stock.pdf}{自然死亡係数($M$)の変化に伴う資源量、親魚量および加入尾数の変化}{fig_M_stock}
{fig/reproduction_Blimit.pdf}{再生産関係とBlimit(Bblimit)の設定}{fig_reproduction_Blimit}

\TwoOfSixFigs
{fig/spawner_recruit.pdf}{親魚量と加入尾数の経年変化}{fig_spawner_recruit}
{fig/RPS.pdf}{RPSの経年変化}{fig_RPS}

\TwoOfSixFigs
{fig/F_SPR_YPR.pdf}{漁獲係数($F$)と\%SPR(実線)およびYPR(破線)との関係}{fig_F_SPR_YPR}
{fig/stock_F.pdf}{資源量と漁獲係数($F$)との関係}{fig_stock_F}

\OneOfSixFigs
{fig/F_SPR_YPR.pdf}{漁獲係数($F$)と\%SPR(実線)およびYPR(破線)との関係}{fig_F_SPR_YPR2}


\TwoOfEightFigs
{fig/maaji_bunpu.pdf}{a}{a}
{fig/maaji_gyojou.pdf}{a}{b}
\TwoOfEightFigs
{fig/maaji_daichu.pdf}{a}{c}
{fig/maaji_gyokaku.pdf}{a}{d}
\TwoOfEightFigs
{fig/maaji_index.pdf}{a}{e}
{fig/maaji_nenrei.pdf}{a}{f}
\TwoOfEightFigs
{fig/maaji_nenreibetu.pdf}{a}{g}
{fig/maaji_seijuku.pdf}{a}{h}
%                   図を読み込み
%\input{_tables}
\setcounter{chapter}{0}

%\include{sweave_test}
%\onecolumn
%\begin{multicols}{2}
%\listoftables
%\listoffigures
%\end{multicols}
%
%
%
\end{document}
