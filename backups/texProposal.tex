\documentclass[12pt]{myjsarticle}
\usepackage{amsmath,amssymb}
\usepackage[dvipdfmx]{graphicx}
\usepackage{float}
%
% ここに自分専用のマクロや \setlength や \newtheorem などを書く。
%
\begin{document}

\title{資源評価業務の円滑化を目的とした\\組版ソフト導入に関する提案書}
\author{資源海洋部 浮魚資源グループ 林 晃}
\date{\today}
\maketitle % 上記のタイトル、名前、年月日を印刷
%\tableofcontents % 目次を印刷
\input{honbun}
% ここに色々書く。

% 以下は文献表。引用するときは \cite{chosha1} のように書く。
%
%\begin{thebibliography}{99}

%\bibitem{chosha1}
%著者1, 論文タイトル1, 雑誌名, Vol.~12, No.~2, 1999, 23--45

%\bibitem{chosha2}
%著者2, 論文タイトル2, 雑誌名, Vol.~104, No.~4, 2000, 223--256

%\end{thebibliography}

\end{document}
