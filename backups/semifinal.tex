\chapter{\thisyrjp(\thisyrad)年度カタクチイワシ対馬暖流系群の資源評価}

\begin{table}[h]
\begin{tabular}{{rp{12truecm}}}
責任担当水研: &西海区水産研究所 (安田十也、林 晃、黒田啓行、髙橋素光)\\
参画機関: & 日本海区水産研究所、青森県産業技術センター水産総合研究所、秋田県水産振興センター、山形県水産試験場、新潟県水産海洋研究所、富山県農林水産総合技術センター水産研究所、石川県水産総合センター、福井県水産試験場、京都府農林水産技術センター海洋センター、兵庫県立農林水産技術総合センター但馬水産技術センター、鳥取県水産試験場、島根県水産技術センター、山口県水産研究センター、福岡県水産海洋技術センター、佐賀県玄海水産振興センター、長崎県総合水産試験場、熊本県水産研究センター、鹿児島県水産技術開発センター
\end{tabular}
\end{table}

\begin{center}\Large{{\gt 要   約}}\end{center}
本系群の資源量について、コホート解析により計算した。
資源量は1995年から2000年まで200千トン以上であったが、2001年に130千トンへ減少した。
2004年以降資源量は増加し、2007年には247千トンとなったが、それ以降減少傾向を示した。
2015年における資源量は132千トンと推定され、前年(120千トン)より増加した。
過去の資源量と親魚量から資源水準は低位、過去5年間(2011~2015年)の資源量の推移から動向は横ばいと判断した。
再生産関係から、Blimitを2005年水準の親魚量91千トンとした。
2015年の親魚量(61千トン)は$Blimit$を下回っている。%$マークではさむと数式モードがオンに
5年後に親魚量を$Blimit$まで回復させる$F$($Frec5yr$)を管理基準値として、2017年ABCを算出した。
\input{_youyaku_table}%表を読み込み
ただし、
本報告での
ABCは仔魚
(シラス)
を
含む
日本の
漁獲に対する値である。%単発の改行は無視される。

Targetは、資源変動の可能性やデータ誤差に起因する評価の不確実性を考慮し、%1行あけると改行される
より安定的な資源の増大が期待される漁獲量である。
Limitは、管理基準の下で許容される最大レベルの漁獲量である。
$Ftarget = \alpha Flimit$とし、係数$\alpha$には標準値0.8を用いた。
漁獲割合は、漁獲量÷資源量とした。$F$値は各年齢の平均とした。
2015年の親魚量は61千トン。
ABCはシラスの漁獲量を含む。
$Frec5yr$は5年後に親魚量を$Blimit$まで回復させるF値。

ただし、$F$は各年齢の単純平均。
シラスの漁獲量を含む。
\thisyrad 年の資源量・親魚量は加入尾数を仮定した値。%今年を呼び出し。

\input{_maegaki}
\subsection{生態}                 %読み込み専用。
\input{output/bunpu_kaiyu}
\input{output/nenrei_seichou}
\input{output/seijuku_sanran}
\input{output/hihoshokukankei}


%\subsection{漁業の状況}
%\subsubsection{漁業の概要}
%\subsubsection{漁獲量の推移}

\subsection{資源の状態}
\subsubsection{資源評価の方法}
\subsubsection{資源量指標値の推移}
\subsubsection{漁獲物の年齢組成}
\subsubsection{資源の水準・動向}
\subsubsection{資源と漁獲の関係}

\subsection{\thisyrad 年ABCの算定}
\subsubsection{資源評価のまとめ}
\subsubsection{ABC並びに推定漁獲量の算定}
\subsubsection{ABClimitの評価}
\subsubsection{ABCの再評価}

\subsection{ABC以外の管理方策の提言}
\subsection{引用文献}
\clearpage
\input{_figures}
\setcounter{chapter}{0}
