\chapter*{★kotoshi(★今年)年度カタクチイワシ対馬暖流系群の資源評価}

\begin{table}[h]
\begin{tabular}{{rp{12truecm}}}
責任担当水研: &西海区水産研究所 (安田十也、林 晃、黒田啓行、髙橋素光)\\
参画機関: & 日本海区水産研究所、青森県産業技術センター水産総合研究所、秋田県水産振興センター、山形県水産試験場、新潟県水産海洋研究所、富山県農林水産総合技術センター水産研究所、石川県水産総合センター、福井県水産試験場、京都府農林水産技術センター海洋センター、兵庫県立農林水産技術総合センター但馬水産技術センター、鳥取県水産試験場、島根県水産技術センター、山口県水産研究センター、福岡県水産海洋技術センター、佐賀県玄海水産振興センター、長崎県総合水産試験場、熊本県水産研究センター、鹿児島県水産技術開発センター
\end{tabular}
\end{table}

\begin{center}\Large{{\gt 要   約}}\end{center}
本系群の資源量について、コホート解析により計算した。資源量は1995年から2000年まで200千トン以上であったが、2001年に130千トンへ減少した。2004年以降資源量は増加し、2007年には247千トンとなったが、それ以降減少傾向を示した。2015年における資源量は132千トンと推定され、前年(120千トン)より増加した。過去の資源量と親魚量から資源水準は低位、過去5年間(2011~2015年)の資源量の推移から動向は横ばいと判断した。再生産関係から、Blimitを2005年水準の親魚量91千トンとした。2015年の親魚量(61千トン)はBlimitを下回っている。5年後に親魚量をBlimitまで回復させるF(Frec5yr)を管理基準値として、2017年ABCを算出した。★表を挿入youyakutable表を挿入★ただし、本報告でのABCは仔魚(シラス)を含む日本の漁獲に対する値である。★改行Targetは、資源変動の可能性やデータ誤差に起因する評価の不確実性を考慮し、より安定的な資源の増大が期待される漁獲量である。Limitは、管理基準の下で許容される最大レベルの漁獲量である。Ftarget = α Flimitとし、係数αには標準値0.8を用いた。漁獲割合は、漁獲量÷資源量とした。F値は各年齢の平均とした。2015年の親魚量は61千トン。ABCはシラスの漁獲量を含む。Frec5yrは5年後に親魚量をBlimitまで回復させるF値。★改行ただし、Fは各年齢の単純平均。シラスの漁獲量を含む。★kotoshi年の資源量・親魚量は加入尾数を仮定した値。

\subsection{まえがき}
我が国周辺に分布するカタクチイワシは、太平洋系群、瀬戸内海系群および対馬暖流系群から構成される。本種の漁獲量は、マイワシとは対照的に1990年代に増加した。対馬暖流域においても、1990年代後半にかけて漁獲量が増加したが、2001年に急減し、その後は増減を繰り返している。しかし、本種の漁獲量の変動幅はマイワシほど大きくない。これは、マイワシと比較して親魚になるまでの期間が短いことや、ほぼ周年にわたり産卵を行うことなどが要因と考えられる。★改行東シナ海や日本海に分布するカタクチイワシは、韓国や中国によっても漁獲されているが、これらの主な分布域は韓国と中国の沿岸域であるため、対馬暖流系群とはみなさず、本資源評価では考慮しなかった。

\subsection{生態}
\subsubsection{分布・回遊}
カタクチイワシは、日本海では日本、朝鮮半島、沿海州の沿岸域を中心に分布する(落合・田中 1986)。過去には、日本海の中央部や間宮海峡以南の北西部でも分布が確認されている(ベリャーエフ・シェルシェンコフ 未発表)。東シナ海では、日本、朝鮮半島、中国の沿岸域を中心にして、沖合域にも分布することが報告されている(図1、Iversen et al. 1993、 Ohshimo 1996)。日本漁船の主漁場は日本海西部および九州北~西岸の沿岸域である。日本海および東シナ海におけるカタクチイワシの詳細な回遊経路は不明である。卵の出現状況からみて、対馬暖流域の産卵は、主に春から夏にかけて対馬暖流の影響下にある水域で行われ、能登半島以南の水域ではさらに秋季まで継続すると考えられる(内田・道津 1958)。
\subsubsection{年齢・成長}
本系群の成長様式は、発生時期によって異なることが知られている。本報告では、耳石に形成される日周輪の解析結果および体長組成の経月変化から、孵化した個体が半年後には被鱗体長で約9 cmまで成長すると仮定した。体長組成の経月変化から、春季と秋季の発生群について成長様式を求めたところ、次のような結果を得た(図2、大下 2009)。★改行★中央揃え春季発生群: $BL_t = 143.96(1-e^{-0.15(t+0.44)})$ ★改行秋季発生群: $BL_t = 158.59(1-e^{-0.09(t+0.74)})$中央揃え★ただし、$BL_t$は孵化後$t$ヶ月の被鱗体長(mm)である。寿命は3年程度と考えられている。
\subsubsection{成熟・産卵}
カタクチイワシは、厳冬期を除いて周年にわたり産卵することが知られている。若狭湾では体長8.5 cmで産卵することが報告されている(Funamoto et al. 2004)。鳥取県沿岸においては、体長11.9 cm以上であれば、ほとんどが産卵すると報告されている(志村ほか 2008)。これらの結果に従えば、春季発生群は翌年の産卵期にほぼ全て産卵することとなる。そのため、本報告では満1歳から全個体が産卵に参加すると仮定した(図3)。
\subsubsection{非捕食関係}
カタクチイワシは、動物プランクトンのうち主にカイアシ類を餌料とする(Tanaka et al. 2006)。本種は多様な動物種の餌料となっており、仔稚魚期にはマアジ・マサバなどの魚食性魚類や肉食性動物プランクトンに、未成魚・成魚期には魚食性魚類の他に、クジラやイルカなどの海産ほ乳類や海鳥類などにも捕食される。

\subsection{漁業の状況}
\subsubsection{漁業の概要}
\subsubsection{漁獲量の推移}

\subsection{資源の状態}
\subsubsection{資源評価の方法}
\subsubsection{資源量指標値の推移}
\subsubsection{漁獲物の年齢組成}
\subsubsection{資源の水準・動向}
\subsubsection{資源と漁獲の関係}

\subsection{★kotoshi年ABCの算定}
\subsubsection{資源評価のまとめ}
\subsubsection{ABC並びに推定漁獲量の算定}
\subsubsection{ABClimitの評価}
\subsubsection{ABCの再評価}

\subsection{ABC以外の管理方策の提言}
\subsection{引用文献}
