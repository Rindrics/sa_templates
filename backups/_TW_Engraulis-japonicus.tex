% このファイルは編集は不要。
\chapter{\thisyrjp(\thisyrad)年度カタクチイワシ対馬暖流系群の資源評価}% 今年の年数を「jp(和暦)」と「ad(西暦)」で呼び出し。
%
\input{output/TW_Engraulis-japonicus_sekinintantou}
{\要約}
本系群の資源量について、コホート解析により計算した。
資源量は1995年から2000年まで200千トン以上であったが、
2001年に\makeatletter\@nameuse{Biomass_thousandton_2001}\makeatother へ減少した。
2004年以降資源量は増加し、2007年には\makeatletter\@nameuse{Biomass_thousandton_2007}\makeatother
となったが、それ以降減少傾向を示した。
{\ThisYr}年における資源量は\makeatletter\@nameuse{Biomass_thousandton_THISYEAR}\makeatother と推定され、
前年(\makeatletter\@nameuse{Biomass_thousandton_LASTYEAR}\makeatother)より
%%%%%%%%
%%%%%%%%
増加した。%%
%%%%%%%%
%%%%%%%%
過去の資源量と親魚量から資源水準は{\StockLevel}、
過去5年間(\makeatletter\@nameuse{YearMinus5}\makeatother \UTF{FF5E}{\LastYr}年)の資源量の推移から動向は{\StockTrend}と判断した。
再生産関係から、Blimitを2005年水準の親魚量\makeatletter\@nameuse{Blimit_thousandton}\makeatother とした。
{\ThisYr}年の親魚量(\makeatletter\@nameuse{SSB_thousandton_THISYEAR}\makeatother)はBlimitを
%%%%%%%%%%
%%%%%%%%%%
下回っている。%%
%%%%%%%%%%
%%%%%%%%%%
5年後に親魚量をBlimitまで回復させるF(Frec5yr)を管理基準値として、{\ABCYr}年ABCを算出した。
ただし、本報告でのABCは仔魚(シラス)を含む日本の漁獲に対する値である。

\begin{center}
\begin{tabularx}{14.1cm}{cccccc}
\toprule
\multirow{2}{*}{管理基準}	& {Target/Limit} 	& {$F$} 	& {漁獲割合(\%)} 	& {\shortstack{\\\ABCYr 年ABC\\(千トン)}} 	& Blimit = 91\newline(千トン)\tabularnewline \cline{6-6}
						& 					& 		& 					& 						& 親魚量5年後(千トン)\tabularnewline
\hline
\multirow{2}{*}{$Frec5yr$}& Target 			& 1.55 	& 44 				& 47 					& 222				\tabularnewline \cline{2-6}
						& Limit				& 1.94	& 48				& 51					& ~91				\tabularnewline
\bottomrule
\end{tabularx}
\end{center}


Targetは、資源変動の可能性やデータ誤差に起因する評価の不確実性を考慮し、
より安定的な資源の増大が期待される漁獲量である。
Limitは、管理基準の下で許容される最大レベルの漁獲量である。
Ftarget = $\alpha$ Flimitとし、係数$\alpha$には標準値{\Alpha}を用いた。
漁獲割合は、漁獲量÷資源量とした。F値は各年齢の平均とした。
{\ThisYr}年の親魚量は\makeatletter\@nameuse{SSB_thousandton_THISYEAR}\makeatother。
ABCはシラスの漁獲量を含む。
Frec5yrは5年後に親魚量をBlimitまで回復させるF。
\過去五年間の資源量等{2012 & 106 & 56 & 55 & 2.21 & 51}{2013 & 101 & 71 & 52 & 2.10 & 52}{2014 & 120 & 78 & 64 & 3.14 & 54}{2015 & 132 & 61 & 66 & 2.48 & 50}{2016 & 131 & 67 & -- & -- & --}


ただし、Fは各年齢の単純平均。
シラスの漁獲量を含む。
{\NextYr}年の資源量・親魚量は加入尾数を仮定した値。
%        要約ファイルを読み込み。以下同じなので略。
\input{output/TW_Engraulis-japonicus_maegaki}
\subsection{分布・回遊} 
マアジ太平洋系群の分布域を図1に、主な漁場形成の模式図を図2に示した。
日本近海のうち太平洋および隣接海域に分布するマアジには、東シナ海を主産卵場とする群と本州中部以南で産卵する地先群があると考えられている。
太平洋沿岸の中部以東の海域では加入時期の異なる群が見られ、2~4月に東シナ海で生まれたものと5月以降に太平洋沿岸域で生まれたものが分布すると考えられている(木幡 1972)。
また、東シナ海からの加入群(横田・三田 1958)の多寡が資源水準を左右するとも考えられている(古藤 1990)。
我が国近海のマアジ資源は東シナ海に本系群と対馬暖流系群共通の産卵場があると考えられるため、両系群あわせて評価することも想定されるが、本系群の親魚が東シナ海に産卵回遊する情報もないため、結論は得られていない。

\subsubsection{年齢・成長}
1年で尾叉長18cm、2年で24cm程度に成長する(図3)。
寿命は5歳前後と考えられるが、4歳魚以上の漁獲は少ない。

\subsubsection{成熟・産卵}
産卵期は南部ほど早く、豊後水道、紀伊水道外域などでは冬から初夏であり(阪本ほか 1986、薬師寺 2001、阪地2001)、相模湾では春から初夏(木幡 1972、澤田 1974)である。
1歳で50%、2歳以上で100%が成熟する(図4)。

\subsubsection{被捕食関係}
仔稚魚は成長するにつれて大型の動物プランクトンを摂餌し、幼魚以降では魚食性が強くなる(三谷ほか 2001)。
本種は大型の魚類等により捕食される(三谷ほか 2001)。

\subsection{漁業の状況}
\input{output/gyogyou_gaiyou}
\input{output/gyokakuryou_suii}

\subsection{資源の状態}
\input{output/shigenhyouka_houhou}
\input{output/shigenryou_shihyouchi}
\input{output/gyokakubutsu_nenreisosei}
\input{output/shigenryou_gyokakuwariai}
\input{output/blimit_settei}
\input{output/shigen_suijun_doukou}
\input{output/shigen_gyokaku_kankei}

\subsection{\thisyrad 年ABCの算定}
\input{output/shigenhyouka_matome}
\input{output/abc_santei}
\input{output/abc_hyouka}
\input{output/abc_saihyouka}

\input{output/TW_Engraulis-japonicus_abc_igai_no_kanrihousaku}

\bibliographystyle{mynatbib}%       文献の引用スタイルを指定。ほんとは総元締めファイルに書きたいが、魚種ごとに文献リストが必要なので仕方ない。
\bibliography{output/TW_Engraulis-japonicus_references}%         引用文献がここに出る。
\clearpage%                         改ページ
\TwoOfSixFigs
{fig/bunpu.pdf}{カタクチイワシ対馬暖流系群の分布域}{fig_bunpu}
{fig/seichou.pdf}{カタクチイワシの成長様式\newline◯: 春季発生群観測値、■: 秋季発生群観測値、△: 年齢別体重、実線: 春季発生群成長式、破線: 秋季発生群成長式。}{fig_seichou}

\TwoOfSixFigs
{fig/mature_age.pdf}{年齢別成熟率}{fig_mature_age}
{fig/catch_anchovy_shirasu.pdf}{カタクチイワシとシラスの漁獲量}{fig_catch_anchovy_shirasu}

\TwoOfSixFigs
{fig/production.pdf}{産卵量の経年変化}{fig_production}
{fig/stock_index.pdf}{現存量指標値}{fig_stock_index}

\TwoOfSixFigs
{fig/shirasuCPUE_kyushu.pdf}{九州北西岸におけるシラス調査CPUE}{fig_shirasuCPUE_kyushu}
{fig/shirasuCPUE_ECS.pdf}{東シナ海におけるシラス調査CPUE}{fig_shirasuCPUE_ECS}

\TwoOfSixFigs
{fig/catch_age.pdf}{年齢別漁獲尾数}{fig_catch_age}
{fig/stock_percentage.pdf}{推定された資源量と漁獲割合}{fig_stock_percentage}

\TwoOfSixFigs
{fig/M_stock.pdf}{自然死亡係数($M$)の変化に伴う資源量、親魚量および加入尾数の変化}{fig_M_stock}
{fig/reproduction_Blimit.pdf}{再生産関係とBlimit(Bblimit)の設定}{fig_reproduction_Blimit}

\TwoOfSixFigs
{fig/spawner_recruit.pdf}{親魚量と加入尾数の経年変化}{fig_spawner_recruit}
{fig/RPS.pdf}{RPSの経年変化}{fig_RPS}

\TwoOfSixFigs
{fig/F_SPR_YPR.pdf}{漁獲係数($F$)と\%SPR(実線)およびYPR(破線)との関係}{fig_F_SPR_YPR}
{fig/stock_F.pdf}{資源量と漁獲係数($F$)との関係}{fig_stock_F}

\OneOfSixFigs
{fig/F_SPR_YPR.pdf}{漁獲係数($F$)と\%SPR(実線)およびYPR(破線)との関係}{fig_F_SPR_YPR2}


\TwoOfEightFigs
{fig/maaji_bunpu.pdf}{a}{a}
{fig/maaji_gyojou.pdf}{a}{b}
\TwoOfEightFigs
{fig/maaji_daichu.pdf}{a}{c}
{fig/maaji_gyokaku.pdf}{a}{d}
\TwoOfEightFigs
{fig/maaji_index.pdf}{a}{e}
{fig/maaji_nenrei.pdf}{a}{f}
\TwoOfEightFigs
{fig/maaji_nenreibetu.pdf}{a}{g}
{fig/maaji_seijuku.pdf}{a}{h}
%                   図を読み込み
%\input{_tables}
\setcounter{chapter}{0}
