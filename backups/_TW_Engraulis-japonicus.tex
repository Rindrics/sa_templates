% このファイルは編集は不要。
\chapter{\thisyrjp(\thisyrad)年度カタクチイワシ対馬暖流系群の資源評価}% 今年の年数を「jp(和暦)」と「ad(西暦)」で呼び出し。
%
\担当機関等
{西海区水産研究所}%担当水研
{安田十也、林 晃、黒田啓行、\CID{8705}橋素光}%担当者
{日本海区水産研究所、青森県産業技術センター水産総合研究所、秋田県水産振興センター、山形県水産試験場、新潟県水産海洋研究所、富山県農林水産総合技術センター水産研究所、石川県水産総合センター、福井県水産試験場、京都府農林水産技術センター海洋センター、兵庫県立農林水産技術総合センター但馬水産技術センター、鳥取県水産試験場、島根県水産技術センター、山口県水産研究センター、福岡県水産海洋技術センター、佐賀県玄海水産振興センター、長崎県総合水産試験場、熊本県水産研究センター、鹿児島県水産技術開発センター}%関連機関

\要約
本系群の資源量について、コホート解析により計算した。
資源量は1995年から2000年まで200千トン以上であったが、2001年に130千トンへ減少した。
2004年以降資源量は増加し、2007年には247千トンとなったが、それ以降減少傾向を示した。
2015年における資源量は132千トンと推定され、前年(120千トン)より増加した。
過去の資源量と親魚量から資源水準は低位、過去5年間(2011~2015年)の資源量の推移から動向は横ばいと判断した。
再生産関係から、Blimitを2005年水準の親魚量91千トンとした。
2015年の親魚量(61千トン)はBlimitを下回っている。
5年後に親魚量をBlimitまで回復させる$F$($Frec5yr$)を管理基準値として、2017年ABCを算出した。%$マークではさむと数式モードがオンに
ただし、
本報告での
ABCは仔魚
(シラス)
を
含む
日本の
漁獲に対する値である。%単発の改行は無視される。

\begin{center}
\begin{tabularx}{14.1cm}{cccccc}
\toprule
\multirow{2}{*}{管理基準}	& {Target/Limit} 	& {$F$} 	& {漁獲割合(\%)} 	& {\shortstack{\\\thisyrad 年ABC\\(千トン)}} 	& Blimit = 91\newline(千トン)\tabularnewline \cline{6-6}
						& 					& 		& 					& 						& 親魚量5年後(千トン)\tabularnewline
\hline
\multirow{2}{*}{$Frec5yr$}& Target 			& 1.55 	& 44 				& 47 					& 222				\tabularnewline \cline{2-6}
						& Limit				& 1.94	& 48				& 51					& ~91				\tabularnewline
\bottomrule
\end{tabularx}
\end{center}
%表を読み込み

Targetは、資源変動の可能性やデータ誤差に起因する評価の不確実性を考慮し、
より安定的な資源の増大が期待される漁獲量である。
Limitは、管理基準の下で許容される最大レベルの漁獲量である。
$Ftarget = \alpha Flimit$とし、係数$\alpha$には標準値0.8を用いた。
漁獲割合は、漁獲量÷資源量とした。$F$値は各年齢の平均とした。
2015年の親魚量は61千トン。
ABCはシラスの漁獲量を含む。
$Frec5yr$は5年後に親魚量をBlimitまで回復させる$F$。
\過去五年間の資源量等{2012 & 106 & 56 & 55 & 2.21 & 51}{2013 & 101 & 71 & 52 & 2.10 & 52}{2014 & 120 & 78 & 64 & 3.14 & 54}{2015 & 132 & 61 & 66 & 2.48 & 50}{2016 & 131 & 67 & -- & -- & --}


ただし、$F$は各年齢の単純平均。
シラスの漁獲量を含む。
\thisyrad 年の資源量・親魚量は加入尾数を仮定した値。%今年を呼び出し。
%        要約ファイルを読み込み。以下同じなので略。
\subsection{まえがき}
我が国周辺に分布するカタクチイワシは、太平洋系群、瀬戸内海系群および対馬暖流系群から構成される。
本種の漁獲量は、マイワシとは対照的に1990年代に増加した。対馬暖流域においても、1990年代後半にかけて漁獲量が増加したが、
2001年に急減し、その後は増減を繰り返している。しかし、本種の漁獲量の変動幅はマイワシほど大きくない。
これは、マイワシと比較して親魚になるまでの期間が短いことや、ほぼ周年にわたり産卵を行うことなどが要因と考えられる。

東シナ海や日本海に分布するカタクチイワシは、韓国や中国によっても漁獲されているが、これらの主な分布域は韓国と中国の沿岸域であるため、
対馬暖流系群とはみなさず、本資源評価では考慮しなかった。

\subsection{分布・回遊} 
マアジ太平洋系群の分布域を図1に、主な漁場形成の模式図を図2に示した。
日本近海のうち太平洋および隣接海域に分布するマアジには、東シナ海を主産卵場とする群と本州中部以南で産卵する地先群があると考えられている。
太平洋沿岸の中部以東の海域では加入時期の異なる群が見られ、2~4月に東シナ海で生まれたものと5月以降に太平洋沿岸域で生まれたものが分布すると考えられている(木幡 1972)。
また、東シナ海からの加入群(横田・三田 1958)の多寡が資源水準を左右するとも考えられている(古藤 1990)。
我が国近海のマアジ資源は東シナ海に本系群と対馬暖流系群共通の産卵場があると考えられるため、両系群あわせて評価することも想定されるが、本系群の親魚が東シナ海に産卵回遊する情報もないため、結論は得られていない。

\subsubsection{年齢・成長}
1年で尾叉長18cm、2年で24cm程度に成長する(図3)。
寿命は5歳前後と考えられるが、4歳魚以上の漁獲は少ない。

\subsubsection{成熟・産卵}
産卵期は南部ほど早く、豊後水道、紀伊水道外域などでは冬から初夏であり(阪本ほか 1986、薬師寺 2001、阪地2001)、相模湾では春から初夏(木幡 1972、澤田 1974)である。
1歳で50%、2歳以上で100%が成熟する(図4)。

\subsubsection{被捕食関係}
仔稚魚は成長するにつれて大型の動物プランクトンを摂餌し、幼魚以降では魚食性が強くなる(三谷ほか 2001)。
本種は大型の魚類等により捕食される(三谷ほか 2001)。

\subsection{漁業の状況}
\input{output/gyogyou_gaiyou}
\input{output/gyokakuryou_suii}

\subsection{資源の状態}
\input{output/shigenhyouka_houhou}
\input{output/shigenryou_shihyouchi}
\input{output/gyokakubutsu_nenreisosei}
\input{output/shigenryou_gyokakuwariai}
\input{output/blimit_settei}
\input{output/shigen_suijun_doukou}
\input{output/shigen_gyokaku}

\subsection{\thisyrad 年ABCの算定}
\input{output/shigenhyouka_matome}
\input{output/abc_santei}
\input{output/abc_hyouka}
\input{output/abc_saihyouka}

\subsection{ABC以外の管理方策の提言}
本種は寿命が短く、漁獲物の大半は0歳魚である。
親魚量と加入尾数には正の相関が見られることから、資源を安定して利用するためには、親魚量を一定以上に保つことが有効である。
そのため、加入が少ないと判断された場合には、0歳魚を獲り控えることが効果的と考えられる。


\bibliographystyle{mynatbib}%       文献の引用スタイルを指定。ほんとは総元締めファイルに書きたいが、魚種ごとに文献リストが必要なので仕方ない。
\bibliography{output/TW_Engraulis-japonicus_references}%         引用文献がここに出る。
\clearpage%                         改ページ
\TwoOfSixFigs
{fig/bunpu.pdf}{カタクチイワシ対馬暖流系群の分布域}{fig_bunpu}
{fig/seichou.pdf}{カタクチイワシの成長様式\newline◯: 春季発生群観測値、■: 秋季発生群観測値、△: 年齢別体重、実線: 春季発生群成長式、破線: 秋季発生群成長式。}{fig_seichou}

\TwoOfSixFigs
{fig/mature_age.pdf}{年齢別成熟率}{fig_mature_age}
{fig/catch_anchovy_shirasu.pdf}{カタクチイワシとシラスの漁獲量}{fig_catch_anchovy_shirasu}

\TwoOfSixFigs
{fig/production.pdf}{産卵量の経年変化}{fig_production}
{fig/stock_index.pdf}{現存量指標値}{fig_stock_index}

\TwoOfSixFigs
{fig/shirasuCPUE_kyushu.pdf}{九州北西岸におけるシラス調査CPUE}{fig_shirasuCPUE_kyushu}
{fig/shirasuCPUE_ECS.pdf}{東シナ海におけるシラス調査CPUE}{fig_shirasuCPUE_ECS}

\TwoOfSixFigs
{fig/catch_age.pdf}{年齢別漁獲尾数}{fig_catch_age}
{fig/stock_percentage.pdf}{推定された資源量と漁獲割合}{fig_stock_percentage}

\TwoOfSixFigs
{fig/M_stock.pdf}{自然死亡係数($M$)の変化に伴う資源量、親魚量および加入尾数の変化}{fig_M_stock}
{fig/reproduction_Blimit.pdf}{再生産関係とBlimit(Bblimit)の設定}{fig_reproduction_Blimit}

\TwoOfSixFigs
{fig/spawner_recruit.pdf}{親魚量と加入尾数の経年変化}{fig_spawner_recruit}
{fig/RPS.pdf}{RPSの経年変化}{fig_RPS}

\TwoOfSixFigs
{fig/F_SPR_YPR.pdf}{漁獲係数($F$)と\%SPR(実線)およびYPR(破線)との関係}{fig_F_SPR_YPR}
{fig/stock_F.pdf}{資源量と漁獲係数($F$)との関係}{fig_stock_F}

\OneOfSixFigs
{fig/F_SPR_YPR.pdf}{漁獲係数($F$)と\%SPR(実線)およびYPR(破線)との関係}{fig_F_SPR_YPR2}


\TwoOfEightFigs
{fig/maaji_bunpu.pdf}{a}{a}
{fig/maaji_gyojou.pdf}{a}{b}
\TwoOfEightFigs
{fig/maaji_daichu.pdf}{a}{c}
{fig/maaji_gyokaku.pdf}{a}{d}
\TwoOfEightFigs
{fig/maaji_index.pdf}{a}{e}
{fig/maaji_nenrei.pdf}{a}{f}
\TwoOfEightFigs
{fig/maaji_nenreibetu.pdf}{a}{g}
{fig/maaji_seijuku.pdf}{a}{h}
%                   図を読み込み
%\input{_tables}
\setcounter{chapter}{0}
