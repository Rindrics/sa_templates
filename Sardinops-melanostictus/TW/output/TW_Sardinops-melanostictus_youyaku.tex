\要約
本系群の資源量について、資源量指標値を考慮したコホート解析により求めた。資源量は、1970年代から増加し、1988年には1千万トンに達したと推定されるが、1990年代に急減した。2001\UTF{FF5E}2003年に過去最低の水準で推移し、2004年以降は増加傾向にある。2015年の資源量は298千トンで、親魚量は192千トンである。2015年の親魚量がBlimit(100千トン)を上回っていることから資源水準は中位で、最近5年間(2011\UTF{FF5E}2015年)の資源量の推移から動向は横ばいと判断した。今後、再生産成功率(加入量÷親魚量)が、不確実性の高い直近年(2015年)を除く過去10年(2005\UTF{FF5E}2014年)の中央値で継続した場合に、現状の漁獲圧の維持(Fcurrent)、親魚量の増大(F40\%SPR)および親魚量の維持(Fmed≒F30\%SPR)の各漁獲シナリオで期待される漁獲量を2017年ABCとして算定した。


%\begin{center}
\begin{tabularx}{14.1cm}{cccccc}
\toprule
\multirow{2}{*}{管理基準}	& {Target/Limit} 	& {$F$} 	& {漁獲割合(\%)} 	& {\shortstack{\\\thisyrad 年ABC\\(千トン)}} 	& Blimit = 91\newline(千トン)\tabularnewline \cline{6-6}
						& 					& 		& 					& 						& 親魚量5年後(千トン)\tabularnewline
\hline
\multirow{2}{*}{$Frec5yr$}& Target 			& 1.55 	& 44 				& 47 					& 222				\tabularnewline \cline{2-6}
						& Limit				& 1.94	& 48				& 51					& ~91				\tabularnewline
\bottomrule
\end{tabularx}
\end{center}


Targetは、資源変動の可能性やデータ誤差に起因する評価の不確実性を考慮し、
より安定的な資源の増大が期待される漁獲量である。
Limitは、管理基準の下で許容される最大レベルの漁獲量である。
$Ftarget = \alpha Flimit$とし、係数$\alpha$には標準値0.8を用いた。
漁獲割合は、漁獲量÷資源量とした。$F$値は各年齢の平均とした。
2015年の親魚量は61千トン。
ABCはシラスの漁獲量を含む。
$Frec5yr$は5年後に親魚量をBlimitまで回復させる$F$。
\過去五年間の資源量等{2012 & 106 & 56 & 55 & 2.21 & 51}{2013 & 101 & 71 & 52 & 2.10 & 52}{2014 & 120 & 78 & 64 & 3.14 & 54}{2015 & 132 & 61 & 66 & 2.48 & 50}{2016 & 131 & 67 & -- & -- & --}


ただし、$F$は各年齢の単純平均。
シラスの漁獲量を含む。
\thisyrad 年の資源量・親魚量は加入尾数を仮定した値。%今年を呼び出し。
