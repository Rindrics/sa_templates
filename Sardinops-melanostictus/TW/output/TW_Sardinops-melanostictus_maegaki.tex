\subsection{まえがき}
我が国周辺に分布するマイワシは、対馬暖流系群と太平洋系群から構成され、1980年代後半に日本周辺域で最も多獲された魚種であり、1988年には日本全体で約450万トンの漁獲量があった。対馬暖流域でも1980年代半ばから1990年代前半にかけて100万トンを超える漁獲量があったが、その後減少し、2000年には1万トンを下回った(表1)。漁獲量の減少要因として、1980年代後半に加入量が連続的に減少したことや、資源の減少に伴って漁獲圧が高くなったことが考えられる。連続した加入の失敗は、人為的な影響ではなく、自然環境的な要因によるものと考えられている(\citealt{Watanabe1995, Ohshimo2009FO})。
マイワシは数十年規模の資源変動をすることが知られるが、再生産関係を考慮し、不適な環境においてもある程度の加入量が見込める親魚量を確保することが重要である。低水準となった場合には、親魚量を増加させ、将来の好適な海洋環境下での加入量の回復に備える必要がある。
。
