\subsubsection{年齢・成長}
\digest{本系群の成長様式は、発生時期によって異なることが知られている。
本報告では、耳石に形成される日周輪の解析結果および体長組成の経月変化から、孵化した個体が半年後には被鱗体長で約9 cmまで成長すると仮定した。}
体長組成の経月変化から、春季と秋季の発生群について成長様式を求めたところ、次のような結果を得た(図\ref{fig_seichou}、\citealt{Ohshimo2009NBJP})。

\begin{center}
春季発生群: $BL_t = 143.96(1-e^{-0.15(t+0.44)})$\\
秋季発生群: $BL_t = 158.59(1-e^{-0.09(t+0.74)})$
\end{center}
ただし、$BL_t$は孵化後$t$ヶ月の被鱗体長(mm)である。
寿命は3年程度と考えられている。
