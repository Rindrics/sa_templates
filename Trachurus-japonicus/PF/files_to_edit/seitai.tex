\subsection{分布・回遊} 
マアジ太平洋系群の分布域を図1に、主な漁場形成の模式図を図2に示した。
日本近海のうち太平洋および隣接海域に分布するマアジには、東シナ海を主産卵場とする群と本州中部以南で産卵する地先群があると考えられている。
太平洋沿岸の中部以東の海域では加入時期の異なる群が見られ、2~4月に東シナ海で生まれたものと5月以降に太平洋沿岸域で生まれたものが分布すると考えられている(木幡 1972)。
また、東シナ海からの加入群(横田・三田 1958)の多寡が資源水準を左右するとも考えられている(古藤 1990)。
我が国近海のマアジ資源は東シナ海に本系群と対馬暖流系群共通の産卵場があると考えられるため、両系群あわせて評価することも想定されるが、本系群の親魚が東シナ海に産卵回遊する情報もないため、結論は得られていない。

\subsubsection{年齢・成長}
1年で尾叉長18cm、2年で24cm程度に成長する(図3)。
寿命は5歳前後と考えられるが、4歳魚以上の漁獲は少ない。

\subsubsection{成熟・産卵}
産卵期は南部ほど早く、豊後水道、紀伊水道外域などでは冬から初夏であり(阪本ほか 1986、薬師寺 2001、阪地2001)、相模湾では春から初夏(木幡 1972、澤田 1974)である。
1歳で50%、2歳以上で100%が成熟する(図4)。

\subsubsection{被捕食関係}
仔稚魚は成長するにつれて大型の動物プランクトンを摂餌し、幼魚以降では魚食性が強くなる(三谷ほか 2001)。
本種は大型の魚類等により捕食される(三谷ほか 2001)。
