\subsection{要約}
 本系群の資源量について、資源量指標値を考慮したコホート解析により計算した。
資源量は1980年代に増加し、1990年代半ばは14万トンから16万トンで推移し高位水準であったが、1997年からは減少に転じ、2006年以降は10万トンを下回る水準となった。
2017年の資源量は\makeatletter\@nameuse{Biomass_thousandton_THISYEAR}\makeatotherである。

親魚量は\makeatletter\@nameuse{SSB_thousandton_THISYEAR}\makeatotherでBlimit(1986年の親魚量\makeatletter\@nameuse{SSB_thousandton_1986}\makeatother)以上であることから資源水準は{\StocktLevel}中位と判断される。
最近5年間(\makeatletter\@nameuse{YearMinus5}\makeatother \UTF{FF5E}{\LastYr})の資源量の推移から動向は{\StockTrend}と判断した。
2017年親魚量はBlimitを上回っているもののその程度はわずかであり、2018年以降は再びBlimit以下の低位水準に減少すると予測される。
したがって2019年のABCは、親魚量をBlimit以上に維持することを管理目標とし、今後、再生産成功率(RPS=加入量/親魚量)が直近年を除く過去5年間(2012~2016年)の平均値で継続した場合に、親魚量の増大の漁獲シナリオで期待される漁獲量として算定した。
2018年、2019年の値は将来予測に基づいた推定値である。Fは各年齢の平均値。
水準:中位  動向:減少
