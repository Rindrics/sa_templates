\subsubsection{今後の加入量の見積もり}
北西太平洋において、小型浮魚類の資源は、気候変動に伴って数十年規模で周期的かつ劇的な変動を繰り返してきた。
例えば、太平洋十年規模変動指数(PDOindex)が正偏差の期間はマイワシ、負偏差の期間はカタクチイワシの資源が高水準となる魚種交替が知られている。
マアジの資源変動様式は、カタクチイワシと相似しており、マイワシと逆の関係にある(Takasuka et al. 2008)。
将来予測においては親魚量とRPSを用い加入量を推定した。
加入量は減少傾向にあり、親魚量はBlimit(24千トン)をわずかに上回っている程度である。
図\ref{fig:R}、\ref{fig:RPS}に示すようにRPSも減少傾向にある。
将来予測に用いるRPS値は、昨年度は過去10年(2007~2016年)のうち下位5年の平均値としたが、今年度は2016年のRPSは19.2尾/kg、2017年が15.8尾/kgとやや増加したことから、近年の低水準期の平均的な値として直近年を除く過去5年間(2012~2016年)のRPSの平均値(14.3尾/kg)を用いた。


