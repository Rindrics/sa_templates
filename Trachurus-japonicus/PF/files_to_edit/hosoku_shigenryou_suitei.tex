\subsection{補足資料 資源量推定}
コホート解析により年齢別資源尾数、資源量、漁獲係数を推定した。
マアジの生活史に基づき1月を起点とした。
使用した生物学的パラメータは図3、図4の通りである。
解析結果は0歳~3+歳(3歳以上をまとめて3+(プラスグループ)と表記する)の年齢別に求めた(補足表2-1)。
年齢別資源尾数Nの計算にはPope(1972)の近似式を用い、プラスグループの資源尾数については平松(1999)の方法を用いた。
自然死亡係数は、田内・田中の式(田中 1960)に従い$M=2.5÷寿命(寿命5歳)$より0.5とした。

1982~2017年までの36年間について、年別年齢別漁獲尾数$C_{a,y}$から、a歳、y年の資源尾数$N_{a,y}$、漁獲係数$Fa,y$は、それぞれ以下の式で求めた。

\begin{equation}
 N_{a,y} = N_{a+1,y+1}\exp{(M)} + C_{a,y}\exp{\left(\frac{M}{2}\right)} (a=0,1, y=1982,…,Y-1)\label{eq:normalN}
\end{equation}

\begin{equation}
 F_{a,y} = -\ln{1-\frac{C_{a,y}\exp{\left(\frac{M}{2}\right)}}{N_{a,y}}} (a = 0, 1, 2, y = 1982, …, Y-1)\label{eq:F}
\end{equation}

 ここで、Yは最近年の2017年を示す。
 3歳以上はプラスグループとし、2歳と3+歳の漁獲係数は等しいと仮定し、資源尾数は以下の式で求めた。

\begin{equation}
 N_{2,y} = \frac{C_{2,y}}{C_{2,y}+ C_{3+,y}} N_{3+,y+1} + C_{2,y}\exp{\left(\frac{M}{2}\right)}(y=1982,…,Y-1)\label{eq:plus-1}
\end{equation} 

\begin{equation}
 N_{3,y} = \frac{C_{3,y}}{C_{2,y}+ C_{3+,y}} N_{3+,y+1} + C_{3,y}\exp{\left(\frac{M}{2}\right)}(y=1982,…,Y-1)\label{eq:plus-1}
\end{equation} 

最近年Yの資源尾数は、
\begin{equation}
 N_{a,y}=\frac{C_{a,Y}}{1-\exp{(-F_{a,Y})}} \exp{\left(\frac{M}{2}\right)} (a=0,…,3+)\label{eq:recent}
\end{equation}
で求めた。

2017年の漁獲係数は、補足表2-4に示した各加入量に関する指標値を用いて、最近年最高齢の$F_{3+,Y}$をチューニングにより推定した。
y年における対数変換したj番目(j = 1,…,6)の加入量指標値の観測値$\ln{I_{j,y}}$と加入量指標値の計算値の残差を最小にする$F_{3+,Y}$を最小二乗法で推定した。

\begin{equation}
 ln(\hat{I_{i,j}}= \ln{q_{j}}N_{0,y} \label{eq:tuningformula}
\end{equation}

\begin{equation}
 RSS = \sum_{j=1}^6 \sum_{y=2005}^Y \left( \ln{\hat{I_{j,y}}} - \ln{I_{j,y}}  \right) ^2 \label{eq:tuningOBJ}
\end{equation}
ここで、qjは漁具能率で以下の式により計算した。

\begin{equation}
 q_{j} = \exp{ \left( \frac{1}{n} \left( \sum_{y=2005}^Y \ln{ \frac{I_{j,y}}{N_y}} \right) \right)} \label{eq:catchabilityCF}
\end{equation}
また、2017年の0~2歳の漁獲係数は、その選択率が過去5年の選択率$s_{a,y}$の平均に等しいと仮定し、以下の式で推定した。

\begin{equation}
 F_{a,Y} = \frac{ \frac{1}{5} \sum_{y=Y-5}^{Y-1} s_{a,y}}{ \frac{1}{5}\sum_{y=Y-5}^{Y-1} s_{3+,y}}F_{3+,Y} (a=0,…,2) \label{eq:terminalF}
\end{equation}

\begin{equation}
 S_{a,y} = \frac{F_{a,y}}{\max{F_{y}}} \label{eq:selectivity}
\end{equation}

3)将来予測
Fcurrentは直近年を除く過去3年(2015年~2017年)のFの平均値とした。
2018のFはFcurrentであるとした。
また将来予測における選択率にはFcurrentの選択率を用いた。
資源尾数の予測には、以下のコホート解析の前進法を用いた。



\begin{equation}
 N_{a,y} = N_{a-1,y-1}\exp{\left(-F_{a-1,y-1}-M\right)} \label{eq:normaltfuture}
\end{equation}
 \begin{equation}
(a=1,2)	
\end{equation}
\begin{equation} 
 N_{3+,y} = N_{3+,y-1}\exp{\left(-F_{3+,y-1}-M\right)} + N_{2,y-1}\exp{\left(-F_{2,y-1}-M\right)} \label{futureplus}
\end{equation}
将来予測における加入量はRPSと親魚量の積とした。

\begin{equation}
 N_{0,y} = SSB_{y}RPS \label{eq:futureR}
\end{equation}
RPSは決定論的シミュレーションでは直近年を除く過去5年間(2012~2016年)のRPS値の平均値14.3尾/kg、確率論的シミュレーションでは同期間のRPS値をランダムに選択した。
資源量の計算に用いる年齢別体重は、2006~2016年の年齢別体重の平均値とした。
なお、将来予測における親魚量が過去最高の64千トンを超える場合、加入量を計算する際の親魚量は64千トンで一定とした。
漁獲尾数は以下の式により推定した。
\begin{equation}
 C_{a,y} = N_{a,y}\left(1-\exp{(-F_{a,y})}\right)\exp{\left(-\frac{M}{2}\right)} \label{eq:futurecatch}
\end{equation}
以上のすべての計算はMS-Excelおよび統計言語RのパッケージRVPA(市野川・岡村 2014)を用いて行った。
