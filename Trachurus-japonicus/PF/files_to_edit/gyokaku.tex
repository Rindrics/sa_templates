\subsubsection{漁業の概要} 
まき網漁業による漁獲が約60%を占め、定置網による漁獲が約30%でこれに次いでいる。
日向灘、豊後水道、紀伊水道から熊野灘では春から秋までの漁獲が多く、相模湾では春が主体である。
これらの海域では春から0歳魚が、年初から1歳魚が漁獲される。
千葉県以北の海域では1歳魚以上の漁獲が多い。

\subsubsection{漁獲量の推移}
太平洋北区~太平洋南区(北海道太平洋北部~宮崎県)における漁獲量の推移を表1、図\ref{fig:landings}に示した。
漁獲量は1982~1985年までは20千トン以下であったが、1986年に急増して37千トンとなり、1990年以降に再び増加して1994~1997年は70千~80千トンと高い水準で推移した。
1997年以降は減少に転じ、2009年以降は30千トン以下で推移した。
2015~2016年は16千~17千トンと極めて低い水準で推移したが、2017年は7月以降、高知県以西の海域で当歳魚を中心に好調な漁獲が続き、総漁獲量は24千トンと2016、2017年を上回る好漁となった。
2017年の漁況については補足資料3に詳細を示した。1990年代後半(1995~1999年)と2017年の漁獲量を海区別に比較すると太平洋南区では31千トンが13千トンと4割に減少し、太平洋中区では26千トンが7千トンと3割に減少している。
一方で太平洋北区は6千トンが3千トンと減少は5割にとどまり、太平洋の西側での漁獲量の減少が顕著である。
海区別漁獲割合は、1990年代後半は太平洋南区5割、太平洋中区4割、太平洋北区1割であった。2017年は太平洋南区6割、太平洋中区3割、太平洋北区1割であり、太平洋中区で減少した。
なお、本系群の外国漁船による漁獲はない。
図\ref{fig:landings}および表1に示した漁獲量は漁業・養殖業生産統計年報に記載された数値に基づき、太平洋各県に計上されている漁獲量から、大中型まき網漁業漁獲成績報告書により東シナ海で漁獲されたと判定された分(水産庁提供、西水研集計)を差し引いた値を用いた。
なお、今年度新たに、太平洋所属船による日本海での漁獲量が太平洋側の漁獲量として計上されていたことが明らかとなったため、2014~2016年について相当する漁獲量を差し引いた。
これにより2014年は975トン、2015年は1,178トン、2016年は1,826トンが昨年の値から減少した。
1989~2001年にかけては、混獲魚(主にサバ類)の漁獲量が漁業・養殖業生産統計年報においてマアジの漁獲量に計上された分を差し引いた。

\subsubsection{漁獲努力量}
 大型定置網の漁労体数は2000年以降太平洋南区では横ばい、太平洋中区では緩やかな減少傾向で推移している(図\ref{fig:doryoku})。
なお、太平洋北区では2007年以降の統計値が非公表であるため推移は不明である。
まき網のうち太平洋北区で操業する北部太平洋まき網について、漁業情報サービスセンター(JAFIC)が集計した年間有効努力量は、2000~2005年まで減少傾向で、その後低水準、横ばいで推移している(図\ref{fig:doryoku})。
