\subsubsection{資源評価の方法}
1982年以降の年齢別漁獲尾数(図\ref{fig:caa}、補足表2-1)に基づいて、
コホート解析により年齢別資源尾数(補足表2-1)、
資源量(図\ref{fig:biomass}、表1、補足表2-1)、
漁獲係数F(図\ref{fig:Fs}、表1、補足表2-1)を計算した(補足資料1、2)。
資源評価に用いた計算では、昨年度と同様、直近年の選択率は過去5年の選択率の平均に等しいと仮定した。
加入量指標値は、一昨年までは宮崎県~静岡県と分布の西側に情報が偏っていたため、昨年度から千葉県での定置網0歳魚漁獲量を加入量指標として追加し、残差が大きい静岡県伊豆東岸定置網0歳魚漁獲量を除外した。
今年度も同様の加入量指数のうち2005~2017年の数値をチューニングに用いた(補足資料2)。
自然死亡係数Mは寿命(本資源では5歳前後)との関係から0.5とした(Tanaka1986)。
