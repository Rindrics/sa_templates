\subsubsection{資源評価の方法}
1982年以降の年齢別漁獲尾数(図\ref{fig:caa}、補足表2-1)に基づいて、
コホート解析により年齢別資源尾数(補足表2-1)、
資源量(図\ref{fig:biomass}、表1、補足表2-1)、
漁獲係数F(図\ref{fig:Fs}、表1、補足表2-1)を計算した(補足資料1、2)。
資源評価に用いた計算では、昨年度と同様、直近年の選択率は過去5年の選択率の平均に等しいと仮定した。
加入量指標値は、一昨年までは宮崎県~静岡県と分布の西側に情報が偏っていたため、昨年度から千葉県での定置網0歳魚漁獲量を加入量指標として追加し、残差が大きい静岡県伊豆東岸定置網0歳魚漁獲量を除外した。
今年度も同様の加入量指数のうち2005~2017年の数値をチューニングに用いた(補足資料2)。
自然死亡係数Mは寿命(本資源では5歳前後)との関係から0.5とした(Tanaka1986)。

\subsubsection{資源量指標値の推移}
加入量水準の指標値には、0歳魚を漁獲対象とする各県各漁法の6種類のCPUE、漁獲量データを用いた(図\ref{fig:Rindex}、{補足資料2})。
①宮崎県南部定置網アジ仔CPUE:宮崎県南郷漁協の定置網に4~6月に入網するアジ仔銘柄(0歳魚)の漁獲量を、対応する定置網の延べ水揚日数で除した値
②宇和島港まき網ゼンゴCPUE:愛媛県宇和島港に中型まき網により水揚げされるゼンゴ銘柄(0歳魚)CPUE(月当たり漁獲量/水揚げ統数)の4月~翌年3月の合計
③宿毛湾中型まき網ゼンゴ資源量指数:高知県宿毛湾において中型まき網により漁獲されるゼンゴ銘柄(0歳魚)の日別漁獲量/出漁隻数を4月~翌年3月まで累積した値
④串本棒受網0歳魚漁獲量:和歌山県串本においてマアジ0歳魚を対象とする棒受網による5月~6月漁獲量
⑤伊勢湾まめ板漁業0歳魚漁獲量:伊勢湾の愛知県小型びき網漁業(まめ板漁業)による4月~翌年3月の0歳魚漁獲量
⑥千葉県定置網0歳魚漁獲量:千葉県鴨川の沖定置と灘定置、千倉の定置網の10月~翌年3月の月別漁獲量平均値
これら6種類の指標値の傾向をみると、2008年に①、③および④で高い値がみられ、2009年以降は変動を繰り返しつつ全体では減少傾向で推移し、⑥は2010年に高い値がみられたが、他の指標値と同様に近年は減少傾向であった。
2017年は①~④の指数は増加、⑤、⑥の指数は減少しており、東西海域で相反する傾向がみられた(図\ref{fig:Rindex}、{補足資料2})。

\subsubsection{漁獲物の年齢組成}
漁獲の主体は0歳魚と1歳魚である。
2015年は0歳魚の漁獲尾数は1982年以降で最低の6,600万尾であったが、2016年は12,609万尾と倍増した。
2017年の0歳魚漁獲尾数は16,101万尾と2016年をさらに上回り、特に宮崎県~高知県において豊漁であった。
しかし1990年代~2000年代前半と比較すると低い水準にある(図\ref{fig:caa}、補足表2-1)。

\subsubsection{資源量と漁獲割合の推移}
資源量は1982年から1990年代始めにかけて増加し、1990年には高位水準になったが、
1996年の162千トンを頂点として減少した(図\ref{fig:biomass},表1)。
その後、2000年と2001年は増加したものの、2004年以降は再び減少した。
2017年の資源量は43千トンと推定された。
親魚量は1984年以降増加し1992年に最高の64千トンとなった。
1993~2000年まで50千トン前後で推移した後、2001~2008年にかけて減少し、2009~2012年は25千~29千トンと横ばいで推移したが、2013、2014年はそれぞれ35千トン、31千トンと増加した。
しかし、2015年は25千トンに減少し、2016年は23千トンとBlimit以下に減少した。
2017年の親魚量は24千トンと推定された。
加入量(0歳資源尾数)は1993年に24億尾と最大になった後は減少傾向にあり、2017年の加入量は3.8億尾であった(図\ref{fig:R}、表1)。
RPSの経年変化をみると1986年($RPS = 7.0尾/kg$)や1993年($RPS = 61.3尾/kg$)と非常に高い値の年があるが
2011年まで平均29.1尾/kgと横ばいで推移していた。
2012年以降は20尾/kgを下回る低い水準となり、とくに2013~2015年は12~13尾/kgと極めて低い水準であった。
2017年は15.8尾/kgであった(図\ref{fig:R})。
自然死亡係数Mを0.4、0.6とした場合の資源量、親魚量について図12に示した。
Mの値が高いほど、いずれの推定値も増加した。
漁獲割合は33~54%の範囲で推移している(図\ref{fig:biomass}、表1)。
各年齢を単純平均した漁獲係数F(Fbar)は0.66~1.60で推移している。
0歳に対するFは総じて1歳以上より相対的に低く、1歳以上に対するFが下がる年にやや上昇する傾向を示している(図\ref{fig:Fs}、補足表2-1)。
2017年のFbar(Fの全年齢平均値)は1.30と推定された(表1)。
資源量とFの間には弱い正の関係がみられる(図\ref{fig:B-F})。

\subsubsection{再生産関係}
親魚量と加入量に正の相関関係が認められ(p0.01、図\ref{fig:S-R})、持続的な資源利用のために親魚量を一定以上に維持することは有効と考えられる。
親魚の回遊経路などに不明な点は多いが、太平洋各地先での親魚量を十分確保する観点から、本系群ではこの再生産関係の仮定のもとに、親魚量を指標とした管理を提案する。

\subsubsection{Blimitの設定}   図\ref{fig:S-R}に示した親魚量と加入量の関係に基づき、それ未満では資源の回復措置をとる閾値(Blimit)は、少ない親魚量から比較的高い加入量が発生した1986年水準の親魚量(24千トン)とした。

\subsection{資源の水準・動向}
2017年の推定資源量は{43千トン}、親魚量は{24千トン}であった。資源水準の基準として、中位と低位の境界は、Blimitとの対応から親魚量{24千トン}(Blimit、1986年の親魚量)とする。
中位と高位の境界は、親魚量の最低~最高値の三等分により{47千トン}とする。
{2017年}の親魚量は{24千トン}と推定され、Blimitをわずかに上回ったことから、{2017年}の資源水準は{中位}にあると判断される。
資源水準が中位に移行した要因の一つとして、{2016年}の再生産成功率が近年では比較的高く、{2016年}の加入量が良好であったことがあげられる。
動向は過去5年間({2013~2017年})の資源量の推移から減少と判断した。


\subsubsection{今後の加入量の見積もり}
北西太平洋において、小型浮魚類の資源は、気候変動に伴って数十年規模で周期的かつ劇的な変動を繰り返してきた。
例えば、太平洋十年規模変動指数(PDOindex)が正偏差の期間はマイワシ、負偏差の期間はカタクチイワシの資源が高水準となる魚種交替が知られている。
マアジの資源変動様式は、カタクチイワシと相似しており、マイワシと逆の関係にある(Takasuka et al. 2008)。
将来予測においては親魚量とRPSを用い加入量を推定した。
加入量は減少傾向にあり、親魚量はBlimit(24千トン)をわずかに上回っている程度である。
図\ref{fig:R}、\ref{fig:RPS}に示すようにRPSも減少傾向にある。
将来予測に用いるRPS値は、昨年度は過去10年(2007~2016年)のうち下位5年の平均値としたが、今年度は2016年のRPSは19.2尾/kg、2017年が15.8尾/kgとやや増加したことから、近年の低水準期の平均的な値として直近年を除く過去5年間(2012~2016年)のRPSの平均値(14.3尾/kg)を用いた。

\subsubsection{生物学的管理基準と現状の漁獲圧の関係}
 前項で設定したRPS平均値で親魚量水準を維持するF(Fsus)は0.66(各年齢の単純平均)と推定された。
SPR並びにYPRの関係(図16)から検討すると、FcurrentはF30%SPR、F0.1、加入量当たり漁獲量を最大化する漁獲係数(Fmax)などの経験的管理基準値より大きく、漁獲圧の削減が必要と考えられる。