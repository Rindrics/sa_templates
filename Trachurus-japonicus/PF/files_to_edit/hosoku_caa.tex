\subsubsection{年齢別漁獲尾数}
 年別年齢別漁獲尾数は、太平洋側の各都県試験研究機関が調査した各都県主要港の水揚量と体長組成を用い算出した。
太平洋側を高知県以西、徳島県・和歌山県、三重県・愛知県、静岡県~東京都、千葉県以北の5区に分割し、各区内の主要港の水揚量と体長組成から月毎に体長階級別漁獲尾数を求めた。
2013年以降は千葉県以北での県による主要漁法の違いを考慮し、まき網主体の千葉県~茨城県と、定置網や底びき網主体の福島県以北とにさらに分割した。
体長階級別漁獲尾数は、補足表2-3に示す月別の年齢と尾叉長の関係を基本とし切断法により年齢別漁獲尾数に変換した。
このように算出した主要港の年齢別漁獲尾数の比率を漁業養殖業生産統計年報の太平洋南区、中区、北区の合計の漁獲量(属人統計)から東シナ海および日本海での漁獲量を差し引いた値に合うように引き延ばして系群全体の年齢別漁獲尾数を算出した(図7)。
なお、切断法で年齢分解が困難な3歳以上はプラスグループとして一括して取り扱った。

