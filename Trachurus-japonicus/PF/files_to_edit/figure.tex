
%\
%{マアジ太平洋系群の分布・回遊図}
%\
%{生活史と漁場形成模式図}
%\
%{年齢と成長の関係}	
%\
%{年齢と成熟率の関係}
%
\Figs
{fig/PF-Trachurus-japonicus_landing.pdf}{漁獲量の経年変化(漁業養殖業生産統計年報)}{fig:landings}
\Figs
{fig/PF-Trachurus-japonicus_Efforts.pdf}{大型定置網の漁労体数と北部まき網の有効努力量の推移}{fig:doryoku}
\Figs
{fig/PF-Trachurus-japonicus_caa.pdf}{年齢別漁獲尾数の経年変化}{fig:caa}
\Figs
{fig/PF-Trachurus-japonicus_B-SSB.pdf}{資源量、親魚量、漁獲割合の経年変化 水準判断の境界(親魚量を指標とする)を点線で記入}{fig:biomass}
\Figs
{fig/PF-Trachurus-japonicus_Fs.pdf}{年齢別漁獲係数と各年齢の単純平均値(Fbar)の経年変化}{fig:Fs}
\Figs
{fig/PF-Trachurus-japonicus_index.pdf}{加入量の各指標値の経年変化}{fig:Rindex}
\Figs
{fig/PF-Trachurus-japonicus_RRPS.pdf}{加入尾数と再生産成功率(RPS)の経年変化}{fig:R}
\Figs
{fig/PF-Trachurus-japonicus_defferent_M.pdf}{自然死亡係数を0.4並びに0.6とした場合の2017年の資源量・親魚量 本評価では0.5を用いた}{fig:M}
\Figs
{fig/PF-Trachurus-japonicus_B-F.pdf}{資源量と漁獲係数(各年齢のF値の単純平均)の関係 白丸は1982~2016年、黒丸は2017年}{fig:B-F}
\Figs
{fig/PF-Trachurus-japonicus_S-R.pdf}{親魚量と加入量の関係(再生産関係) 白丸は1982~2016年、黒丸は2017年、破線 はBlimitの(1986年)親魚量}{fig:S-R}
\Figs
{fig/PF-Trachurus-japonicus_S-RPS.pdf}{親魚量と再生産成功率の関係 白丸は1982~2016年、黒丸は2017年}{fig:RPS}
\Figs
{fig/PF-Trachurus-japonicus_YPRSPR.pdf}{漁獲係数F(単純平均)とYPRおよびSPRの関係}{fig:SPRYPR}


%{fig/PF-Trachurus-japonicus_Scenario.pdf}
%{さまざまなFによる漁獲量、資源量、親魚量の将来予測

%
%
%0.8F40%SPR{fig/PF-Trachurus-japonicus_scenarios0.8F40.SPR.pdf} 
%F40%SPR
%{fig/PF-Trachurus-japonicus_scenariosF40.SPR.pdf}     
%0.8・0.75Fsus
%{fig/PF-Trachurus-japonicus_scenarios0.8Fsus0.75.pdf} 
%0.75Fsus
%{fig/PF-Trachurus-japonicus_scenariosFsus0.75.pdf}   
%0.8F30%SPR
%{fig/PF-Trachurus-japonicus_scenarios0.8F30.SPR.pdf}  
%F30%SPR
%{fig/PF-Trachurus-japonicus_scenariosF30.SPR.pdf} 
%0.8Fcurrent
%{fig/PF-Trachurus-japonicus_scenarios0.8Fcurrent.pdf}
%Fcurrent
%{fig/PF-Trachurus-japonicus_scenariosFcurrent.pdf}%

%0.8Fsus
%{fig/PF-Trachurus-japonicus_scenarios0.8Fsus.pdf}    
%Fsus
%{fig/PF-Trachurus-japonicus_scenariosFsus.pdf}        
%\
%%{各漁獲シナリオでの、不確実性を考慮した1,000回のシミュレーションによる漁獲量、資源量、親魚量の将来予測 太い実線は平均値、破線は上下側10%(80%区間)、親魚量の点線はBlimit、細い実%線は1,000回中任意の10回の試行例。

%\
%{0歳魚Fの削減率と漁獲量、資源量、親魚量の変化
%\end{comment}
