\subsubsection{資源量と漁獲割合の推移}
資源量は1982年から1990年代始めにかけて増加し、1990年には高位水準になったが、
1996年の162千トンを頂点として減少した(図\ref{fig:biomass},表1)。
その後、2000年と2001年は増加したものの、2004年以降は再び減少した。
2017年の資源量は43千トンと推定された。
親魚量は1984年以降増加し1992年に最高の64千トンとなった。
1993~2000年まで50千トン前後で推移した後、2001~2008年にかけて減少し、2009~2012年は25千~29千トンと横ばいで推移したが、2013、2014年はそれぞれ35千トン、31千トンと増加した。
しかし、2015年は25千トンに減少し、2016年は23千トンとBlimit以下に減少した。
2017年の親魚量は24千トンと推定された。
加入量(0歳資源尾数)は1993年に24億尾と最大になった後は減少傾向にあり、2017年の加入量は3.8億尾であった(図\ref{fig:R}、表1)。
RPSの経年変化をみると1986年($RPS = 7.0尾/kg$)や1993年($RPS = 61.3尾/kg$)と非常に高い値の年があるが
2011年まで平均29.1尾/kgと横ばいで推移していた。
2012年以降は20尾/kgを下回る低い水準となり、とくに2013~2015年は12~13尾/kgと極めて低い水準であった。
2017年は15.8尾/kgであった(図\ref{fig:R})。
自然死亡係数Mを0.4、0.6とした場合の資源量、親魚量について図12に示した。
Mの値が高いほど、いずれの推定値も増加した。
漁獲割合は33~54%の範囲で推移している(図\ref{fig:biomass}、表1)。
各年齢を単純平均した漁獲係数F(Fbar)は0.66~1.60で推移している。
0歳に対するFは総じて1歳以上より相対的に低く、1歳以上に対するFが下がる年にやや上昇する傾向を示している(図\ref{fig:Fs}、補足表2-1)。
2017年のFbar(Fの全年齢平均値)は1.30と推定された(表1)。
資源量とFの間には弱い正の関係がみられる(図\ref{fig:B-F})。

