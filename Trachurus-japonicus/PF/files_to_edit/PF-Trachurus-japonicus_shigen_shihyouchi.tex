\subsubsection{資源量指標値の推移}
加入量水準の指標値には、0歳魚を漁獲対象とする各県各漁法の6種類のCPUE、漁獲量データを用いた(図\ref{fig:Rindex}、{補足資料2})。
①宮崎県南部定置網アジ仔CPUE:宮崎県南郷漁協の定置網に4~6月に入網するアジ仔銘柄(0歳魚)の漁獲量を、対応する定置網の延べ水揚日数で除した値
②宇和島港まき網ゼンゴCPUE:愛媛県宇和島港に中型まき網により水揚げされるゼンゴ銘柄(0歳魚)CPUE(月当たり漁獲量/水揚げ統数)の4月~翌年3月の合計
③宿毛湾中型まき網ゼンゴ資源量指数:高知県宿毛湾において中型まき網により漁獲されるゼンゴ銘柄(0歳魚)の日別漁獲量/出漁隻数を4月~翌年3月まで累積した値
④串本棒受網0歳魚漁獲量:和歌山県串本においてマアジ0歳魚を対象とする棒受網による5月~6月漁獲量
⑤伊勢湾まめ板漁業0歳魚漁獲量:伊勢湾の愛知県小型びき網漁業(まめ板漁業)による4月~翌年3月の0歳魚漁獲量
⑥千葉県定置網0歳魚漁獲量:千葉県鴨川の沖定置と灘定置、千倉の定置網の10月~翌年3月の月別漁獲量平均値
これら6種類の指標値の傾向をみると、2008年に①、③および④で高い値がみられ、2009年以降は変動を繰り返しつつ全体では減少傾向で推移し、⑥は2010年に高い値がみられたが、他の指標値と同様に近年は減少傾向であった。
2017年は①~④の指数は増加、⑤、⑥の指数は減少しており、東西海域で相反する傾向がみられた(図\ref{fig:Rindex}、{補足資料2})。