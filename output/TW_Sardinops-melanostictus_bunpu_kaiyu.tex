\subsubsection{分布・回遊}
東シナ海北部から九州沿岸(西岸)、日本海にかけて広く分布する。漁獲量が多かった1980年代には沖合域にも分布が見られたが(\citealt{Hiyama1998})、2000年以降はほぼ沿岸域に限られると考えられており(図1)、資源量や生息環境の変化とともに分布域が変化すると考えられる。また、マイワシは主に春と秋に多く漁獲される傾向があり、漁獲量には季節変化がみられる。漁獲量のピークを迎える季節は地域によって異なることから、マイワシは分布域内を大小さまざまな規模で季節回遊しているものと考えられる(伊藤 1961, 黒田 1991)。