\TwoOfSixFigs
{fig/bunpu.pdf}{カタクチイワシ対馬暖流系群の分布域}{fig_bunpu}
{fig/seichou.pdf}{カタクチイワシの成長様式\newline◯: 春季発生群観測値、■: 秋季発生群観測値、△: 年齢別体重、実線: 春季発生群成長式、破線: 秋季発生群成長式。}{fig_seichou}

\TwoOfSixFigs
{fig/mature_age.pdf}{年齢別成熟率}{fig_mature_age}
{fig/catch_anchovy_shirasu.pdf}{カタクチイワシとシラスの漁獲量}{fig_catch_anchovy_shirasu}

\TwoOfSixFigs
{fig/production.pdf}{産卵量の経年変化}{fig_production}
{fig/stock_index.pdf}{現存量指標値}{fig_stock_index}

\TwoOfSixFigs
{fig/shirasuCPUE_kyushu.pdf}{九州北西岸におけるシラス調査CPUE}{fig_shirasuCPUE_kyushu}
{fig/shirasuCPUE_ECS.pdf}{東シナ海におけるシラス調査CPUE}{fig_shirasuCPUE_ECS}

\TwoOfSixFigs
{fig/catch_age.pdf}{年齢別漁獲尾数}{fig_catch_age}
{fig/stock_percentage.pdf}{推定された資源量と漁獲割合}{fig_stock_percentage}

\TwoOfSixFigs
{fig/M_stock.pdf}{自然死亡係数($M$)の変化に伴う資源量、親魚量および加入尾数の変化}{fig_M_stock}
{fig/reproduction_Blimit.pdf}{再生産関係とBlimit(Bblimit)の設定}{fig_reproduction_Blimit}

\TwoOfSixFigs
{fig/spawner_recruit.pdf}{親魚量と加入尾数の経年変化}{fig_spawner_recruit}
{fig/RPS.pdf}{RPSの経年変化}{fig_RPS}

\TwoOfSixFigs
{fig/F_SPR_YPR.pdf}{漁獲係数($F$)と\%SPR(実線)およびYPR(破線)との関係}{fig_F_SPR_YPR}
{fig/stock_F.pdf}{資源量と漁獲係数($F$)との関係}{fig_stock_F}

\OneOfSixFigs
{fig/F_SPR_YPR.pdf}{漁獲係数($F$)と\%SPR(実線)およびYPR(破線)との関係}{fig_F_SPR_YPR2}


\TwoOfEightFigs
{fig/maaji_bunpu.pdf}{a}{a}
{fig/maaji_gyojou.pdf}{a}{b}
\TwoOfEightFigs
{fig/maaji_daichu.pdf}{a}{c}
{fig/maaji_gyokaku.pdf}{a}{d}
\TwoOfEightFigs
{fig/maaji_index.pdf}{a}{e}
{fig/maaji_nenrei.pdf}{a}{f}
\TwoOfEightFigs
{fig/maaji_nenreibetu.pdf}{a}{g}
{fig/maaji_seijuku.pdf}{a}{h}
