\要約
本系群の資源量について、資源量指標値を考慮したコホート解析により求めた。
資源量は、1970年代から増加し、1988年には1千万トンに達したと推定されるが、
1990年代に急減した。2001--2003年に過去最低の水準で推移し、2004年以降は増加傾向にある。
2015年の資源量は298千トンで、親魚量は192千トンである。
2015年の親魚量がBlimit(100千トン)を上回っていることから資源水準は中位で、
最近5年間(2011--2015年)の資源量の推移から動向は横ばいと判断した。
今後、再生産成功率(加入量÷親魚量)が、不確実性の高い直近年(2015年)を除く過去10年
(2005--2014年)の中央値で継続した場合に、
現状の漁獲圧の維持(Fcurrent)、親魚量の増大(F40\%SPR)および親魚量の維持(Fmed≒F30\%SPR)の各漁獲シナリオで期待される漁獲量を2017年ABCとして算定した。

%\input{output/youyaku_table}%表を読み込み

Targetは資源変動やデータ誤差に起因する評価の不確実性を考慮し、各漁獲シナリオの下で、より安定的な資源の増大または維持が期待されるF値による漁獲量で、Limitは各漁獲シナリオの下で許容される最大のF値による漁獲量である。Ftarget = α Flimitとし、係数αには標準値0.8を用いた。Fcurrentは2006--2015年のFの平均値、漁獲割合は2017年の漁獲量/資源量、F値は各年齢の平均値である。漁獲シナリオにある「親魚量の維持」は中長期的に安定する親魚量の維持を指す。{\thisyrad}年の親魚量は192千トン。
