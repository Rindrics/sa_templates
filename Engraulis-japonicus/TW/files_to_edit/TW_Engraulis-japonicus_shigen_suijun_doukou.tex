\subsubsection{資源の水準・動向}
Blimitである親魚量(\makeatletter\@nameuse{Blimit_thousandton}\makeatother)を資源水準の「低位」と「中位」の境界とした。
また親魚量の最小値を基準とした場合に、
親魚量の最大値までの増分の上位1/3と2/3の境界(\textcolor[cmyk]{0,1,0,0}{155千トン})を「高位」と「中位」の境界とした。
なお、同様の方法において下位1/3にあたる親魚量は\textcolor[cmyk]{0,1,0,0}{100千トン}で、これはBlimitに比較的近似している。
{\ThisYr}\,年の親魚量(\makeatletter\@nameuse{SSB_thousandton_THISYEAR}\makeatother)がBlimitを下回っていることから、資源の水準を{\StockLevel}と判断した。
動向は、過去5年(\makeatletter\@nameuse{YearMinus5}\makeatother\UTF{FF5E}\makeatletter\@nameuse{YearMinus1}\makeatother\,年)の資源量と親魚量の推移から{\StockTrend}と判断した。
