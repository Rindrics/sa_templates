\subsubsection{Blimitの設定}
親魚量と加入尾数との関係を図\,\ref{fig_reproduction_Blimit}に示した。
親魚量と加入尾数は正の相関を示した。
%
%
RPSの上位10\%と加入尾数の上位10\%にそれぞれ相当する2直線の交点から、
資源回復の閾値となるBlimitを
親魚量\makeatletter\@nameuse{Blimit_thousandton}\makeatother\,
%
%↓要確認↓要確認↓要確認↓要確認↓要確認↓要確認
(2005年水準)とした。
%↑要確認↑要確認↑要確認↑要確認↑要確認↑要確認
%
{\ThisYr}\,年の親魚量は\makeatletter\@nameuse{SSB_thousandton_THISYEAR}\makeatother\,であり、
Blimitを
%
%
%
%↓要確認↓要確認↓要確認↓要確認↓要確認↓要確認			
下回っている。
%↑要確認↑要確認↑要確認↑要確認↑要確認↑要確認
%
%
%
%
%
%
親魚量と加入量の経年変化を図\,\ref{fig_spawner_recruit}に、
RPSの経年変化を図\,\ref{fig_RPS}に示した。
RPSは増減を繰り返しながらも周期的な変化がみられる。
F(各年齢のFの平均値)とYPRおよび\%SPRの関係を図\,\ref{fig_F_SPR_YPR}に示した。
{\ThisYr}\,年のF(\textcolor[cmyk]{0,1,0,0}{2.48})はFmed(\textcolor[cmyk]{0,1,0,0}{2.12})やF30\%SPR(\textcolor[cmyk]{0,1,0,0}{1.29})、Fmax(\textcolor[cmyk]{0,1,0,0}{0.91})、F0.1(\textcolor[cmyk]{0,1,0,0}{0.60})よりも高い。
