\subsubsection{ABCの算定}
本系群では、資源量および再生産関係が明らかとなっており、また親魚量がBlimitを下回っているため、ABC算定ルール1-1)-(2)を用い、
5年後(2021年)に親魚量をBlimitまで回復させる$F$($Frec5yr$)を管理基準値として、2017年ABCを算出した。
ABC算定のための式は次の通りである。
\begin{eqnarray*}
Flimit &=& Frec5yc\\
Ftarget &=& \alpha Flimit
\end{eqnarray*}
Flimitは、5年後(2021年)に親魚量がBlimitまで回復する$F$($Frec5yr$)とし、$\alpha$は基準値の0.8とした。
2016年の$F$は$Fcurrent$($F2015$)とし、
2016年以降の再生産成功率は、直近年を除く過去10年間(2005~2014年)の中央値(777尾/kg)で推移すると仮定した。
また、加入尾数の上限を過去10年間(2006~2015年)の最大値(1,293億尾)と仮定した。
算出したABCは、以下の通りである。
なお、ABCはシラスの漁獲量を含む。

★表を入れる★
★表を入れる★
★表を入れる★

Targetは、資源変動の可能性やデータ誤差に起因する評価の不確実性を考慮し、より安定的な資源の増大が期待される漁獲量である。
Limitは、管理基準の下で許容される最大レベルの漁獲量である。$Ftarget = \alpha Flimit$とし、係数$\alpha$には標準値0.8を用いた。
漁獲割合は、漁獲量÷資源量 $\times100$である。$F$は各年齢の平均である。
