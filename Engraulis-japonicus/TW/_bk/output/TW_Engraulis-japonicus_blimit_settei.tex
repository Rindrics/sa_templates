\subsubsection{Blimitの設定}
親魚量と加入尾数との関係を図\ref{fig_reproduction_Blimit}に示した。
親魚量と加入尾数は正の相関を示した。
RPSの上位10\%と加入尾数の上位10\%にそれぞれ相当する2直線の交点から、
資源回復の閾値となるBlimitを親魚量91千トン(2005年水準)とした。
2015年の親魚量は61千トンであり、Blimitを下回っている。
親魚量と加入量の経年変化を図\ref{fig_spawner_recruit}に、
RPSの経年変化を図\ref{fig_RPS}に示した。
RPSは増減を繰り返しながらも周期的な変化がみられる。
$F$(各年齢の$F$の平均値)とYPRおよび\%SPRの関係を図\ref{fig_F_SPR_YPR}に示した。
2015年の$F$(2.48)は$Fmed$(2.12)や$F$30\%SPR(1.29)、$Fmax$(0.91)、$F$0.1(0.60)よりも高い。
