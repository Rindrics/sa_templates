\subsubsection{ABCの再評価}
2015年(2016年再評価)では、2014年の漁獲量および2015年における年齢別体重を更新した。
また、再生産成功率を本年度評価と同一と仮定し、2019年における親魚量がBlimitへ回復する$F$を求めた。
2016年(2016年再評価)では、再評価時の最近年の資源量推定結果を用いて、2020年における親魚量がBlimitへ回復する$F$を求めた。
資源量推定値は昨年度評価時の値を上回り、やや高めの$F$でも資源回復が可能となったため、2016年のABCはやや多く見積もられた。
この主な要因は、2015年の0歳魚の体重および漁獲尾数が昨年度の予測より大きく、2015年の年齢別体重に基づく将来の親魚量がより多く見積もられたためである。
平成27年度まで本系群の資源評価報告書では、$Frec5yr$を$Frec$と表記していた。
