\subsubsection{資源の水準・動向}
Blimitである親魚量(91千トン)を資源水準の「低位」と「中位」の境界とした。
また親魚量の最小値を基準とした場合に、
親魚量の最大値までの増分の上位1/3と2/3の境界(155千トン)を「高位」と「中位」の境界とした。
なお、同様の方法において下位1/3にあたる親魚量は100千トンで、これはBlimitに比較的近似している。
2015年の親魚量(61千トン)がBlimitを下回っていることから、資源の水準を低位と判断した。
動向は、過去5年(2011年~2015年)の資源量と親魚量の推移から横ばいと判断した。
