\要約
本系群の資源量について、コホート解析により計算した。
資源量は1995年から2000年まで200千トン以上であったが、2001年に130千トンへ減少した。
2004年以降資源量は増加し、2007年には247千トンとなったが、それ以降減少傾向を示した。
2015年における資源量は132千トンと推定され、前年(120千トン)より増加した。
過去の資源量と親魚量から資源水準は低位、過去5年間(2011\UTF{FF5E}2015年)の資源量の推移から動向は横ばいと判断した。
再生産関係から、Blimitを2005年水準の親魚量91千トンとした。
2015年の親魚量(61千トン)はBlimitを下回っている。
5年後に親魚量をBlimitまで回復させる$F$($Frec5yr$)を管理基準値として、2017年ABCを算出した。%$マークではさむと数式モードがオンに
ただし、
本報告での
ABCは仔魚
(シラス)
を
含む
日本の
漁獲に対する値である。%単発の改行は無視される。

\input{output/youyaku_tbl}

Targetは、資源変動の可能性やデータ誤差に起因する評価の不確実性を考慮し、
より安定的な資源の増大が期待される漁獲量である。
Limitは、管理基準の下で許容される最大レベルの漁獲量である。
$Ftarget = \alpha Flimit$とし、係数$\alpha$には標準値0.8を用いた。
漁獲割合は、漁獲量÷資源量とした。$F$値は各年齢の平均とした。
2015年の親魚量は61千トン。
ABCはシラスの漁獲量を含む。
$Frec5yr$は5年後に親魚量をBlimitまで回復させる$F$。
\begin{center}
\begin{tabular}{cccccc}
\toprule
\multirow{2}{*}{年}	& {資源量} 	& {親魚量} 	& {漁獲量}	&	{$F$} & {漁獲割合}\tabularnewline
					& 	(千トン)	& (千トン)		&	(千トン)&		&	(\%)	\tabularnewline
\hline
2012 & 106 & 56 & 55 & 2.21 & 51 \tabularnewline
2013 & 106 & 56 & 55 & 2.21 & 51 \tabularnewline
2014 & 106 & 56 & 55 & 2.21 & 51 \tabularnewline
2015 & 106 & 56 & 55 & 2.21 & 51 \tabularnewline
2016 & 106 & 56 & 55 & 2.21 & 51 \tabularnewline
\bottomrule
\end{tabular}
\end{center}


ただし、$F$は各年齢の単純平均。
シラスの漁獲量を含む。
\thisyrad 年の資源量・親魚量は加入尾数を仮定した値。%今年を呼び出し。
