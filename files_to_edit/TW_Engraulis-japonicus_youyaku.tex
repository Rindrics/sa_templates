{\要約}
本系群の資源量について、コホート解析により計算した。
資源量は1995年から2000年まで200千トン以上であったが、
2001年に\makeatletter\@nameuse{Biomass_thousandton_2001}\makeatother へ減少した。
2004年以降資源量は増加し、2007年には\makeatletter\@nameuse{Biomass_thousandton_2007}\makeatother
となったが、それ以降減少傾向を示した。
{\ThisYr}年における資源量は\makeatletter\@nameuse{Biomass_thousandton_THISYEAR}\makeatother と推定され、
前年(\makeatletter\@nameuse{Biomass_thousandton_LASTYEAR}\makeatother)より
%%%%%%%%
%%%%%%%%
増加した。%%
%%%%%%%%
%%%%%%%%
過去の資源量と親魚量から資源水準は{\StockLevel}、
過去5年間(\makeatletter\@nameuse{YearMinus5}\makeatother \UTF{FF5E}{\LastYr}年)の資源量の推移から動向は{\StockTrend}と判断した。
再生産関係から、Blimitを2005年水準の親魚量\makeatletter\@nameuse{Blimit_thousandton}\makeatother とした。
{\ThisYr}年の親魚量(\makeatletter\@nameuse{SSB_thousandton_THISYEAR}\makeatother)はBlimitを
%%%%%%%%%%
%%%%%%%%%%
下回っている。%%
%%%%%%%%%%
%%%%%%%%%%
5年後に親魚量をBlimitまで回復させるF(Frec5yr)を管理基準値として、{\ABCYr}年ABCを算出した。
ただし、本報告でのABCは仔魚(シラス)を含む日本の漁獲に対する値である。

\begin{center}
\begin{tabularx}{14.1cm}{cccccc}
\toprule
\multirow{2}{*}{管理基準}	& {Target/Limit} 	& {$F$} 	& {漁獲割合(\%)} 	& {\shortstack{\\\ABCYr 年ABC\\(千トン)}} 	& Blimit = 91\newline(千トン)\tabularnewline \cline{6-6}
						& 					& 		& 					& 						& 親魚量5年後(千トン)\tabularnewline
\hline
\multirow{2}{*}{$Frec5yr$}& Target 			& 1.55 	& 44 				& 47 					& 222				\tabularnewline \cline{2-6}
						& Limit				& 1.94	& 48				& 51					& ~91				\tabularnewline
\bottomrule
\end{tabularx}
\end{center}


Targetは、資源変動の可能性やデータ誤差に起因する評価の不確実性を考慮し、
より安定的な資源の増大が期待される漁獲量である。
Limitは、管理基準の下で許容される最大レベルの漁獲量である。
Ftarget = $\alpha$ Flimitとし、係数$\alpha$には標準値{\Alpha}を用いた。
漁獲割合は、漁獲量÷資源量とした。F値は各年齢の平均とした。
{\ThisYr}年の親魚量は\makeatletter\@nameuse{SSB_thousandton_THISYEAR}\makeatother。
ABCはシラスの漁獲量を含む。
Frec5yrは5年後に親魚量をBlimitまで回復させるF。
\過去五年間の資源量等{2012 & 106 & 56 & 55 & 2.21 & 51}{2013 & 101 & 71 & 52 & 2.10 & 52}{2014 & 120 & 78 & 64 & 3.14 & 54}{2015 & 132 & 61 & 66 & 2.48 & 50}{2016 & 131 & 67 & -- & -- & --}


ただし、Fは各年齢の単純平均。
シラスの漁獲量を含む。
{\NextYr}年の資源量・親魚量は加入尾数を仮定した値。
