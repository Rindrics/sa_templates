\documentclass[../master_plan2]{subfiles}
\begin{document}
\subsection{まえがき}
我が国周辺に分布するカタクチイワシは、太平洋系群、瀬戸内海系群および対馬暖流系群から構成される\cite{Hunter1981}。
本種の漁獲量は、マイワシとは対照的に1990年代に増加した。対馬暖流域においても、1990年代後半にかけて漁獲量が増加したが、
2001年に急減し、その後は増減を繰り返している。しかし、本種の漁獲量の変動幅はマイワシほど大きくない。
これは、マイワシと比較して親魚になるまでの期間が短いことや、ほぼ周年にわたり産卵を行うことなどが要因と考えられる。

東シナ海や日本海に分布するカタクチイワシは、韓国や中国によっても漁獲されているが、これらの主な分布域は韓国と中国の沿岸域であるため、
対馬暖流系群とはみなさず、本資源評価では考慮しなかった。
\end{document}
