\subsubsection{資源量と漁獲割合の推移}
コホート解析(補足資料\ref{appendix_stock_method})を用いて、本系群の資源尾数・漁獲係数(表\,\ref{table_stock_F})
及び資源量・親魚量・再生産成功率RPS(加入尾数÷親魚量)・漁獲割合(漁獲量÷資源量 $\times$100)(表\ref{table_stock_etc}、図\,\ref{fig_stock_percentage})を推定した。
1977年以降における資源量の最低値は1979年における\makeatletter\@nameuse{Biomass_thousandton_1979}\makeatother であり、資源量はその後、増減を繰り返しながらも徐々に増加した。
資源量は1998年に\makeatletter\@nameuse{Biomass_thousandton_1998}\makeatother の最大値を記録したが、2001年には\makeatletter\@nameuse{Biomass_thousandton_2001}\makeatother にまで減少した。
資源量はその後、2007年まで再び増加傾向を示したが、2008年以降には減少傾向にある。
{\ThisYr}年の資源量は\makeatletter\@nameuse{Biomass_thousandton_THISYEAR}\makeatother で、前年(\makeatletter\@nameuse{Biomass_thousandton_LASTYEAR}\makeatother )より増加したものの1987年以来の低水準であった。
漁獲割合は、1977年以降\textcolor[cmyk]{0,1,0,0}{50\%}前後で推移し、{\ThisYr}\,年の値は\textcolor[cmyk]{0,1,0,0}{50\%}だった。
 自然死亡係数(M)を0.5、1.0(規定値)、1.5とした場合の資源量・親魚量・加入尾数の推定値を図\,\ref{fig_M_stock}に示した。
資源量は、Mを0.5に仮定した場合には規定値の\textcolor[cmyk]{0,1,0,0}{72\%}となり、$M$を1.5に仮定した場合には\textcolor[cmyk]{0,1,0,0}{144\%}となった。
