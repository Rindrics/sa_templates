\subsubsection{ABCの算定}
本系群では、資源量および再生産関係が明らかとなっており、また親魚量がBlimitを下回っているため、ABC算定ルール1-1)-(2)を用い、
5年後(\makeatletter\@nameuse{YearPlus5}\makeatother\,年)に親魚量をBlimitまで回復させるF(Frec5yr)を管理基準値として、\makeatletter\@nameuse{YearPlus1}\makeatother\,年ABCを算出した。														
ABC算定のための式は次の通りである。
\begin{eqnarray*}
\mathrm{Flimit} &=& \mathrm{Frec5yr}\\
\mathrm{Ftarget} &=& \alpha \mathrm{Flimit}
\end{eqnarray*}
Flimitは、5年後(\makeatletter\@nameuse{YearPlus5}\makeatother\,年)に親魚量がBlimitまで回復するF(Frec5yr)とし、$\alpha$は基準値の{\Alpha}とした。
{\ThisYr}\,年のFはFcurrent(F{\LastYr})とし、
{\ThisYr}\,年以降の再生産成功率は、直近年を除く過去10\,年間(\makeatletter\@nameuse{YearMinus11}\makeatother\,\UTF{FF5E}\makeatletter\@nameuse{YearMinus2}\makeatother\,年)の中央値
(\textcolor[cmyk]{0,1,0,0}{777尾/kg})
で推移すると仮定した。
また、加入尾数の上限を過去10年間(\makeatletter\@nameuse{YearMinus10}\makeatother\,\UTF{FF5E}\makeatletter\@nameuse{YearMinus1}\makeatother\,年)の最大値
(\textcolor[cmyk]{0,1,0,0}{1,293億尾})
と仮定した。
算出したABCは、以下の通りである。
なお、ABCはシラスの漁獲量を含む。

★表を入れる★
★表を入れる★
★表を入れる★

Targetは、資源変動の可能性やデータ誤差に起因する評価の不確実性を考慮し、より安定的な資源の増大が期待される漁獲量である。
Limitは、管理基準の下で許容される最大レベルの漁獲量である。Ftarget = $\alpha$ Flimitとし、係数$\alpha$には標準値{\Alpha}を用いた。
漁獲割合は、漁獲量÷資源量 $\times100$である。Fは各年齢の平均である。
