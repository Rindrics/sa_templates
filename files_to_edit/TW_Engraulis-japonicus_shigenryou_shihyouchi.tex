\subsubsection{資源量指標値の推移}
\textcolor[cmyk]{0,1,0,0}{日本海と東シナ海における産卵量の推移を図\,\ref{fig_production}に示す。
産卵量は1998\UTF{FF5E}2000年に多く、2001年には少なかったものの、2004年には合計10,084兆粒と1979年以降における最大値を示した。
その後、産卵量は増減を繰り返している。
2015年における産卵量の水準は日本海および東シナ海ともに中程度で、合計値は2,471兆粒であった。
%
%
夏季(8・9月)に九州北西岸で行われている、音響調査による現存量指標値\citep{Ohshimo2004}
および中層トロール調査のCPUE(漁獲尾数÷有効網数)を図\,\ref{fig_stock_index}に示す。
現存量指標値は増減を繰り返しながら推移しており、近年では2007年の134.0(相対値)が最も高かった。
現存量指標値はその後、急減し、2010\UTF{FF5E}2012年は2.5\UTF{FF5E}17.9と低水準で推移した。
2013年の現存量指標値は2007年の値の半分を超える程度まで回復し、2015年は108.8となった。
また、中層トロール調査のCPUEは、1990年代後半に比べると、2002年以降は低水準で変動している。
2015年のCPUEは67.4(kg/網)で、前年の値(12.3 kg/網)を大きく上回った。
%
%
九州北西岸で実施した調査において、ニューストンネットに入網したシラスのCPUEの推移を図\,\ref{fig_shirasuCPUE_kyushu}に示した。
6月に実施した調査におけるCPUEは、2003年(598尾/網)、2005年(815尾/網)、2009\UTF{FF5E}2011年(475\UTF{FF5E}928尾/網)に高い値を示したが、
2012年以降には299尾/網以下と低い水準にある。
%
8・9月の調査では、CPUEは2010年から2013年にかけて4\UTF{FF5E}25尾/網と低い水準にあったが、2014年は214尾/網と大きく増加した。
しかし、2015年には67尾/網となり、前年を下回った。
その他主要魚種の採集個体数と、それに対応する有効曳網数は補足資料\ref{appendix_neuston}に示した。
4月に東シナ海で実施した調査において、ニューストンネットに入網したシラスのCPUEの推移を図\,\ref{fig_shirasuCPUE_ECS}に示した。
2003\UTF{FF5E}2007年における値(385\UTF{FF5E}765尾/網)に比べると、2008年\UTF{FF5E}2010年は28\UTF{FF5E}93尾/網と低い水準にあったが、
2011年以降増加傾向を示し、2015年には大幅に増加して1622尾/網となった。2015年は前年を下回り955尾/網であった。}
