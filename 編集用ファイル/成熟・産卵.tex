\subsubsection{成熟・産卵}
カタクチイワシは、厳冬期を除いて周年にわたり産卵することが知られている。
若狭湾では体長8.5 cmで産卵することが報告されている\citep{Funamoto2004}。
鳥取県沿岸においては、体長11.9 cm以上であれば、ほとんどが産卵すると報告されている\citep{Shimura2008}。
これらの結果に従えば、春季発生群は翌年の産卵期にほぼ全て産卵することとなる。
そのため、本報告では満1歳から全個体が産卵に参加すると仮定した(図\ref{fig_mature_age})。
