\subsubsection{漁獲量の推移}
本系群の漁獲量は、漁業・養殖業生産統計年報の青森県~鹿児島県の合計値から、
東シナ海区に所属する漁船による太平洋海域における漁獲量(漁獲成績報告書による)
を差し引いた値とした(表\ref{table_catch}、図\ref{fig_catch_anchovy_shirasu})。
本系群の漁獲量は、1997年を除いて1996年から2000年までは100千トンを超えていたが、
その後2004年には61千トンにまで減少した。
漁獲量は2005年から2008年にかけて再び増加したが、2009年以降は減少傾向にあり、2015年は61千トンであった。
%
%
 海区別では、日本海北区の漁獲量は1995年に9千トンまで増加した後、1996年、2001年、2005年を除いて
 5千トン前後で変動していたが、2011年から2013年にかけて3千トンを下回った(表\ref{table_catch})。
 2015年の漁獲量は3千トンであった。
 %
 %
 %
日本海西区の漁獲量は、1991年から1998年にかけて70千トンまで増加したが、その後減少し、2001年以降は20千トン前後で推移した。
2015年は11千トンと少なかった(表\ref{table_catch})。
東シナ海区の漁獲量は、1990年から2000年(65千トン)まで増加傾向にあった。
その後は、2009年(26千トン)を除いて、2001から40~70千トンで推移しており、2015年は47千トンであった(表\ref{table_catch})。
%
%
対馬暖流域の沿岸域における仔魚(シラス)の漁獲量は、1977年以降1987年まで2千トンから6千トンの間で緩やかに増減したが、
それ以降10年間ほど6千トン前後の漁獲が維持された(表\ref{table_catch})。
漁獲量は1999年と2000年には10千トンを超えたが、2002年にかけて急減した。
漁獲量はその後、2005年前後に再び10千トン近くまで増加したが、2008年以降から減少傾向を示し、2015年には5千トンとなった。
%
%
韓国におけるカタクチイワシ漁獲量は、1995年以降20万トンを超えており、2000年以降は増減を繰り返している
(表\ref{table_catch};水産統計(韓国海洋水産部)、\url{http://www.fips.go.kr:7001/index.jsp}、2016年3月)。
2015年における漁獲量は21万トンであった。
韓国近海の漁場は韓国南岸および東岸である\citep{Korfish2000}。
%
%
中国の漁獲量は、日本・韓国よりも多く、1996年以降50万トン以上で維持されているが、
2003年に約111万トンとなって以降、2009年まで減少が続いた
(FAO Fishery and Aquaculture Statistics. Global capture production 1950--2014、
\url{http://www.fao.org/fishery/statistics/software/fishstatj/en}、2016年6月)。
中国の漁獲量は2009年以降増加しており(表\ref{table_catch})、
データが利用可能な直近年である2014年における値は93万トンであった。
