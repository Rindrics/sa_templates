\subsubsection{分布・回遊}
カタクチイワシは、日本海では日本、朝鮮半島、沿海州の沿岸域を中心に分布する
\citep{Ochiai1986}。
過去には、日本海の中央部や間宮海峡以南の北西部でも分布が確認されている(ベリャーエフ・シェルシェンコフ 未発表)。
東シナ海では、日本、朝鮮半島、中国の沿岸域を中心にして、
沖合域にも分布することが報告されている
(図\ref{fig_bunpu}、\citet{Iversen1993, Ohshimo1996})。
日本漁船の主漁場は日本海西部および九州北~西岸の沿岸域である。
日本海および東シナ海におけるカタクチイワシの詳細な回遊経路は不明である。
卵の出現状況からみて、対馬暖流域の産卵は、主に春から夏にかけて対馬暖流の影響下にある水域で行われ、
能登半島以南の水域ではさらに秋季まで継続すると考えられる\citep{Uchida1958}。
